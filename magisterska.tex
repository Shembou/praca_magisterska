%\documentclass[xodstep,a4paper,twoside,openany,12pt,openright]{wnspt}
\documentclass[a4paper,twoside,12pt]{report}
\usepackage[utf8]{inputenc}
\usepackage[T1]{fontenc}
\usepackage[lmargin=2cm,rmargin=2cm, tmargin=2.5cm,bmargin=2.5cm]{geometry}
\usepackage{amsmath}
\usepackage{amsfonts}
\usepackage{amsthm}
\usepackage{amssymb}
\usepackage{polski}
\usepackage{colortbl}
\usepackage{url}
\usepackage{setspace}
\usepackage{indentfirst}
\usepackage{listingsutf8}
\usepackage{beramono}
\usepackage{graphicx}
\usepackage{hyperref}
\usepackage{enumitem}
\usepackage{xcolor}
\usepackage{float}

% --- opcje ---
\lstdefinestyle{rustcode}{
	language=C++,
	backgroundcolor=\color{gray!10},
	basicstyle=\ttfamily\small,
	keywordstyle=\color{blue}\bfseries,
	stringstyle=\color{red},
	commentstyle=\color{green!50!black},
	numbers=left,
	numberstyle=\tiny\color{gray},
	stepnumber=1,
	numbersep=8pt,
	showstringspaces=false,
	breaklines=true,
	frame=single,
	rulecolor=\color{black!30},
	tabsize=4,
	captionpos=b
}

\newtheorem{lemat}{Lemat}
\newtheorem{twierdzenie}{Twierdzenie}

\renewcommand{\lstlistingname}{Listing}
\renewcommand{\lstlistlistingname}{Spis listingów}

% --- autor ---

\title{Sposoby optymalizacji aplikacji internetowych.

Ways to optimize web applications}


\begin{document}
\thispagestyle{empty}
\begin{center}
UNIWERSYTET JANA DŁUGOSZA W CZĘSTOCHOWIE 
\vspace{2em}\\
\includegraphics[width=0.2\textwidth]{logo_ujd.png}\\
\vspace{2em}
Wydział Nauk Ścisłych, Przyrodniczych i Technicznych
\end{center}

\vspace{2em}
\begin{flushleft}
Kierunek: \textbf{Informatyka}\\
Specjalność: \textbf{Programowanie gier komputerowych}
\end{flushleft}

\vspace{2em}
\begin{center}
\textbf{ \large Przemysław Kamiński}

Nr albumu: \textbf{88751}

\vspace{2em}

{\bfseries \Large 
Sposoby optymalizacji aplikacji internetowych. 

Ways to optimize web applications
}
\end{center}
\vfill
\vspace{2em}
\begin{flushright}
\begin{tabular}{l}
 Praca magisterska \\
 przygotowana pod kierunkiem\\
 dr hab. Bożeny Woźnej-Szcześniak, prof. UJD
\end{tabular}
\end{flushright}
\vfill
\vspace{2em}
\begin{center}
Częstochowa, 2025   
\end{center}

\newpage
\thispagestyle{empty}
\null\newpage
\begin{abstract}
	Celem niniejszej pracy jest porównanie wydajności protokołów komunikacyjnych stosowanych w aplikacjach webowych oraz analiza wpływu formatów wymiany danych na efektywność komunikacji w nowoczesnych systemach informatycznych. Badania obejmują protokoły HTTP/1.1, HTTP/2, GraphQL, gRPC, SOAP, WebSocket Secure oraz WebSocket, a także formaty danych takie jak JSON, XML, Protocol Buffers i MessagePack. Analiza koncentruje się na porównaniu rozmiaru przesyłanych danych, czasu przetwarzania oraz kompatybilności pomiędzy systemami.
	
	Dodatkowo w pracy oceniono wpływ zastosowania technologii WebAssembly na wydajność przetwarzania danych po stronie klienta w porównaniu z tradycyjnym podejściem opartym na języku JavaScript. Eksperymenty zostały przeprowadzone z wykorzystaniem dedykowanej aplikacji testowej, umożliwiającej rzetelny pomiar parametrów wydajnościowych w kontrolowanym środowisku.
	
	Uzyskane wyniki pozwalają określić zależności pomiędzy wyborem protokołów komunikacyjnych, formatów danych oraz technologii wykonawczych a osiąganą wydajnością systemu. Na ich podstawie sformułowano praktyczne wnioski i rekomendacje, które mogą wspierać proces projektowania i optymalizacji nowoczesnych aplikacji webowych oraz architektur rozproszonych.
	
	\begin{center}
		\textbf{Abstract}
	\end{center}
	
	The aim of this thesis is to compare the performance of communication protocols used in web applications and to analyze the impact of data exchange formats on communication efficiency in modern information systems. The study covers protocols such as HTTP/1.1, HTTP/2, GraphQL, gRPC, SOAP, and WebSocket, as well as data formats including JSON, XML, Protocol Buffers, and MessagePack. The analysis focuses on data size, processing time, and cross-system compatibility.
	
	Additionally, the thesis evaluates the impact of using WebAssembly on client-side data processing performance compared to the traditional JavaScript-based approach. The experiments were conducted using a dedicated test application that enabled reliable performance measurements in a controlled environment.
	
	The obtained results allow identifying relationships between the choice of communication protocols, data formats, and execution technologies and the resulting system performance. Based on these findings, practical conclusions and recommendations are formulated to support the design and optimization of modern web applications and distributed architectures.
	
	\noindent\textbf{Słowa kluczowe:}{Rust, WebAssembly, JavaScript, Sieci komputerowe, Protokoły komunikacyjne, Optymalizacja, Serializacja, Deserializacja, Klient--serwer}
\end{abstract}


\onehalfspacing
\thispagestyle{empty}
\tableofcontents


\chapter*{Wstęp}

Dynamiczny rozwój aplikacji webowych oraz rosnące wymagania dotyczące ich skalowalności, responsywności i niezawodności sprawiają, że wydajność komunikacji sieciowej staje się jednym z kluczowych aspektów projektowania nowoczesnych systemów informatycznych. Współczesne aplikacje coraz częściej opierają się na architekturach rozproszonych, w których wymiana danych pomiędzy klientem a serwerem, a także poszczególnymi usługami, odbywa się z wykorzystaniem różnych protokołów komunikacyjnych oraz formatów danych. Wybór odpowiednich technologii w tym zakresie ma bezpośredni wpływ na czas odpowiedzi systemu, obciążenie sieci oraz koszty przetwarzania po obu stronach komunikacji.

Celem niniejszej pracy magisterskiej jest analiza i porównanie wydajności wybranych protokołów komunikacyjnych stosowanych w aplikacjach webowych, a także ocena wpływu formatów wymiany danych oraz technologii wykonywania kodu po stronie klienta na ogólną efektywność systemu. W pracy szczególną uwagę poświęcono porównaniu protokołów HTTP/1.1, HTTP/2, GraphQL, gRPC, SOAP oraz WebSocket, które reprezentują różne podejścia do komunikacji w środowiskach sieciowych. Analizie poddano również popularne formaty danych, takie jak JSON, XML, Protocol Buffers oraz MessagePack, uwzględniając ich rozmiar, czas serializacji i deserializacji oraz kompatybilność pomiędzy różnymi systemami.

Dodatkowym celem pracy jest zbadanie wpływu zastosowania technologii WebAssembly na wydajność przetwarzania danych po stronie klienta w porównaniu z tradycyjnym podejściem opartym na języku JavaScript. W tym kontekście przeprowadzono eksperymenty mające na celu ocenę różnic w czasie wykonywania operacji, zużyciu zasobów oraz potencjalnych korzyściach wynikających z wykorzystania WebAssembly w aplikacjach webowych.

Podsumowując, praca ma na celu dostarczenie praktycznych i eksperymentalnie potwierdzonych wniosków, które mogą stanowić wsparcie przy wyborze protokołów komunikacyjnych, formatów danych oraz technologii wykonawczych w projektowaniu wydajnych i nowoczesnych aplikacji webowych.

\chapter{Wprowadzenie do problematyki doboru technologii}

\section{Znaczenie doboru technologii w aplikacjach webowych}

Współczesne aplikacje webowe stanowią podstawę funkcjonowania wielu systemów informatycznych wykorzystywanych w biznesie, administracji publicznej oraz życiu codziennym. Rosnąca liczba użytkowników, potrzeba obsługi dużych wolumenów danych oraz wymagania dotyczące niskich czasów odpowiedzi powodują, że zagadnienia związane z wydajnością i skalowalnością systemów stają się kluczowe już na etapie projektowania architektury. Jednym z najważniejszych czynników wpływających na te cechy jest dobór odpowiednich technologii komunikacyjnych oraz formatów wymiany danych.

W architekturach rozproszonych komunikacja pomiędzy komponentami systemu odbywa się za pośrednictwem sieci komputerowej, która wprowadza dodatkowe opóźnienia oraz ograniczenia przepustowości. Niewłaściwy wybór protokołu komunikacyjnego lub formatu danych może prowadzić do nadmiernego obciążenia sieci, zwiększonego zużycia zasobów obliczeniowych oraz pogorszenia doświadczeń użytkownika końcowego. Z tego względu świadomy dobór technologii powinien być oparty nie tylko na ich popularności, lecz przede wszystkim na analizie wymagań danego problemu.

\section{Charakterystyka współczesnych systemów rozproszonych}

Nowoczesne systemy informatyczne coraz częściej projektowane są w oparciu o architektury rozproszone\cite{Systemy_rozporszone}, takie jak architektura klient--serwer, mikroserwisy\cite{Microservice} czy systemy oparte na komunikacji zdarzeniowej\cite{event_driven_architecture}. W tego typu rozwiązaniach poszczególne komponenty systemu działają niezależnie i komunikują się ze sobą za pomocą jasno zdefiniowanych interfejsów.

Taki model projektowy umożliwia łatwiejszą skalowalność oraz rozwój systemu, jednak jednocześnie zwiększa znaczenie wydajnej i niezawodnej komunikacji. Każde wywołanie zdalne wiąże się z kosztami transmisji danych, serializacji\cite{serialization_wiki} i deserializacji\cite{deserialization_wiki} komunikatów oraz przetwarzania po obu stronach połączenia. W praktyce oznacza to, że nawet niewielkie różnice w zastosowanych technologiach mogą prowadzić do zauważalnych różnic w wydajności całego systemu.

\section{Protokoły komunikacyjne jako fundament wymiany danych}

Protokoły komunikacyjne definiują sposób, w jaki dane są przesyłane pomiędzy uczestnikami komunikacji. W aplikacjach webowych powszechnie wykorzystywane są protokoły oparte na rodzinie HTTP, jednak wraz z rozwojem technologii pojawiły się również alternatywne rozwiązania, takie jak gRPC, WebSocket, GraphQL. Każdy z tych protokołów oferuje inne właściwości w zakresie wydajności, sposobu zestawiania połączeń, obsługi strumieniowania danych oraz kompatybilności z istniejącą infrastrukturą.

Dobór protokołu komunikacyjnego powinien uwzględniać charakter wymiany danych, częstotliwość komunikacji, wymagania dotyczące opóźnień oraz środowisko uruchomieniowe aplikacji. Przykładowo, protokoły oparte na długotrwałych połączeniach mogą być bardziej efektywne w aplikacjach czasu rzeczywistego, natomiast klasyczne podejście żądanie--odpowiedź sprawdza się w prostszych scenariuszach komunikacyjnych.

\section{Uzasadnienie podjęcia badań}

Różnorodność dostępnych protokołów komunikacyjnych, formatów danych oraz technologii wykonawczych sprawia, że projektanci systemów informatycznych stają przed trudnym zadaniem wyboru najbardziej odpowiednich rozwiązań. Brak jednoznacznych odpowiedzi oraz silne uzależnienie wyników od kontekstu zastosowania powodują, że decyzje te często podejmowane są intuicyjnie.

Celem niniejszej pracy jest dostarczenie empirycznych danych oraz praktycznych wniosków, które mogą wspierać proces podejmowania decyzji technologicznych. Przeprowadzone analizy i eksperymenty pozwalają lepiej zrozumieć zależności pomiędzy wyborem technologii a wydajnością aplikacji webowych.

\section{Słownik pojęć i skrótów}

\begin{description}
	\item[\textbf{API} (\textit{Application Programming Interface})] --
	Zbiór reguł, definicji oraz mechanizmów umożliwiających komunikację pomiędzy różnymi komponentami oprogramowania\cite{api_wiki}.
	
	\item[\textbf{HTTP} (\textit{Hypertext Transfer Protocol})] --
	Protokół komunikacyjny wykorzystywany do przesyłania danych w sieci WWW w modelu klient--serwer.
	
	\item[\textbf{WebSocket}] --
	Protokół umożliwiający utrzymywanie stałego, dwukierunkowego połączenia pomiędzy klientem a serwerem, wykorzystywany do komunikacji w czasie rzeczywistym.
	
	\item[\textbf{gRPC}] --
	Wysokowydajny framework komunikacyjny typu RPC, oparty na protokole HTTP/2 oraz binarnym formacie \textit{Protocol Buffers}.
	
	\item[\textbf{RPC} (\textit{Remote Procedure Call})] --
	Mechanizm komunikacji międzyprocesowej umożliwiający wykonywanie procedur lub funkcji w zdalnym systemie komputerowym tak, jakby były wywoływane lokalnie\cite{RPC}.
	
	\item[\textbf{Protocol Buffers} (\textit{Protobuf})] --
	Binarny format serializacji danych opracowany przez firmę Google, charakteryzujący się kompaktową reprezentacją oraz efektywnym przetwarzaniem, wykorzystywany m.in. w komunikacji gRPC.
	
	\item[\textbf{SOAP} (\textit{Simple Object Access Protocol})] --
	Protokół komunikacyjny oparty na wymianie komunikatów XML, stosowany w architekturach usług sieciowych.
	
	\item[\textbf{MQTT} (\textit{Message Queuing Telemetry Transport})] --
	Lekki protokół komunikacyjny typu publish--subscribe, przeznaczony do systemów o ograniczonych zasobach oraz komunikacji w sieciach o niskiej przepustowości\cite{MQTT}.
	
	\item[\textbf{JSON} (\textit{JavaScript Object Notation})] --
	Tekstowy format wymiany danych oparty na składni JavaScript, charakteryzujący się czytelnością dla człowieka oraz łatwością parsowania przez maszyny.
	
	\item[\textbf{MessagePack}] --
	Binarny format serializacji danych zaprojektowany jako wydajniejsza alternatywa dla JSON, oferujący mniejszy rozmiar oraz szybsze przetwarzanie przy zachowaniu podobnej elastyczności.
	
	\item[\textbf{Apache Avro}] --
	System serializacji danych opracowany w ramach projektu Apache Hadoop, wykorzystujący schematy JSON do definiowania struktury danych oraz zapewniający kompaktową reprezentację binarną\cite{avro}.
	
	\item[\textbf{BSON} (\textit{Binary JSON})] --
	Binarny format serializacji dokumentów wzorowany na JSON, wykorzystywany m.in. w bazie danych MongoDB, rozszerzający JSON o dodatkowe typy danych oraz umożliwiający efektywne przechowywanie i przetwarzanie\cite{BSON}.
	
	\item[\textbf{Serializacja}] --
	Proces przekształcania struktury danych do postaci umożliwiającej jej zapis lub transmisję pomiędzy systemami\cite{serialization_wiki}.
	
	\item[\textbf{Deserializacja}] --
	Proces odtwarzania pierwotnej struktury danych na podstawie jej zserializowanej reprezentacji\cite{deserialization_wiki}.
	
	\item[\textbf{XML} (\textit{eXtensible Markup Language})] --
	Rozszerzalny język znaczników wykorzystywany do opisu i strukturyzacji danych tekstowych.
	
	\item[\textbf{Mikroserwis}] --
	Architektoniczny wzorzec projektowy, w którym aplikacja składa się z wielu niezależnych, luźno powiązanych usług, z których każda realizuje konkretną funkcjonalność biznesową i może być rozwijana oraz wdrażana niezależnie.
	
	\item[\textbf{Systemy rozproszone}] --
	Zbiór niezależnych komponentów obliczeniowych połączonych siecią, które współpracują ze sobą w celu realizacji wspólnego zadania, prezentując się użytkownikowi jako jeden spójny system.
	
	\item[\textbf{Komunikacja zdarzeniowa} (\textit{Event-Driven Architecture})] --
	Architektura oprogramowania, w której komponenty systemu komunikują się poprzez generowanie, wykrywanie i reagowanie na zdarzenia, umożliwiając luźne powiązanie oraz asynchroniczną wymianę informacji.
	
	\item[\textbf{Strumieniowanie danych}] --
	Technika przetwarzania i przesyłania danych w sposób ciągły, bez konieczności oczekiwania na kompletny zbiór danych\cite{stream}.
	
	\item[\textbf{Head-of-Line Blocking} (\textit{HOL})] --
	Zjawisko polegające na blokowaniu przetwarzania kolejnych żądań przez opóźnienie jednego z nich w ramach tego samego połączenia\cite{HOL}.
	
	\item[\textbf{WebAssembly} (\textit{Wasm})] --
	Binarny format wykonywalnego kodu umożliwiający uruchamianie aplikacji o wysokiej wydajności w środowisku przeglądarek internetowych.
	
	\item[\textbf{TLS} (\textit{Transport Layer Security})] --
	Protokół kryptograficzny zapewniający poufność, integralność oraz uwierzytelnianie danych przesyłanych w sieci komputerowej\cite{tls}.
	
	\item[\textbf{Architektura klient--serwer}] --
	Model architektoniczny, w którym klient inicjuje żądania, a serwer przetwarza je i zwraca odpowiedzi\cite{client_server_meaning}.
	
	\item[\textbf{Silnik JavaScript V8}] --
	Wysokowydajny silnik JavaScript opracowany przez firmę Google, wykorzystywany m.in. w przeglądarce Google Chrome oraz środowisku Node.js\cite{Jv8}.
	
	\item[\textbf{DOM} (\textit{Document Object Model})] --
	Model obiektowy dokumentu HTML lub XML, który reprezentuje jego strukturę w postaci drzewa obiektów, umożliwiając programom (np. skryptom JavaScript) dynamiczny dostęp do zawartości, struktury oraz stylów strony internetowej\cite{DOM}.
	
	\item[\textbf{Web Worker}] --
	Mechanizm platformy webowej umożliwiający uruchamianie skryptów JavaScript w tle, w osobnym wątku względem głównego wątku interfejsu użytkownika, co pozwala na wykonywanie kosztownych obliczeń bez blokowania renderowania strony oraz interakcji użytkownika. Web Worker nie ma bezpośredniego dostępu do drzewa DOM, a komunikacja z głównym wątkiem odbywa się za pomocą asynchronicznego przesyłania komunikatów\cite{web_worker}.
	
\end{description}

\chapter{Technologie wykorzystane w projekcie}

\section{Rust}

Rust jest nowoczesnym językiem programowania systemowego, który kładzie duży nacisk na bezpieczeństwo pamięci, wydajność oraz wielowątkowość. Dzięki mechanizmowi własności (ownership) oraz statycznej analizie błędów w czasie kompilacji, Rust minimalizuje ryzyko występowania błędów takich jak wycieki pamięci czy dereferencja pustych wskaźników. Rust zdobył popularność również dzięki nowoczesnemu systemowi typów i możliwości pisania kodu niskopoziomowego bez rezygnacji z bezpieczeństwa \cite{rust}.

Wybór języka Rust był podyktowany potrzebą uzyskania niskich czasów odpowiedzi oraz stabilności działania aplikacji przy dużej liczbie równoległych zapytań. Dodatkowo paczki wykorzystane do badań nie posiadały zbędnych funkcjonalności, co umożliwiło skupienie się na analizie wyników i minimalnej implementacji rozwiązań.

Według Tiobe Index, Rust zyskał największą dotychczas popularność 1 stycznia 2026 roku.

\begin{figure}[h]
	\centering
	\includegraphics[width=0.7\textwidth]{images/rust_tiobe_index}
	\caption{Pozycja języka Rust w rankingu Tiobe Index.}
	\label{fig:rust_tiobe}
\end{figure}

Jak widać na Rysunku~\ref{fig:rust_tiobe}, Rust zyskuje coraz większą popularność wśród programistów, co potwierdza rosnące zainteresowanie nim w przemyśle i projektach open source.

\section{JavaScript}

JavaScript jest językiem skryptowym powszechnie wykorzystywanym do tworzenia aplikacji webowych \cite{javascript}. W projekcie został użyty głównie jako punkt odniesienia dla tradycyjnych interfejsów użytkownika w porównaniu z technologią WebAssembly. Dzięki swojej uniwersalności i dużemu ekosystemowi bibliotek, JavaScript nadal pozostaje podstawowym językiem front-endowym.

\section{TypeScript}

TypeScript jest statycznie typowanym nadzbiorem języka JavaScript, rozwijanym przez firmę Microsoft. Rozszerza on JavaScript o system typów oraz mechanizmy weryfikacji poprawności kodu na etapie kompilacji, co znacząco ułatwia tworzenie, utrzymanie i rozwój większych aplikacji \cite{typesciript}.

\section{Framework Actix-web}

Actix-web jest asynchronicznym frameworkiem webowym dla języka Rust, opartym na modelu aktorów. Charakteryzuje się wysoką wydajnością oraz niskim narzutem czasowym, co sprawia, że jest jedną z najszybszych opcji w ekosystemie Rust \cite{actix}.

Framework Actix-web został wybrany ze względu na:
\begin{itemize}[itemsep=0.2em, parsep=0pt, topsep=0pt]
	\item wysoką wydajność potwierdzoną testami benchmarkowymi,
	\item dobrą integrację z ekosystemem Rust,
	\item wsparcie dla programowania asynchronicznego.
\end{itemize}

\begin{figure}[h]
	\centering
	\includegraphics[width=0.7\textwidth]{images/rust_actix_web}
	\caption{Framework Actix-web w porównaniu z innymi frameworkami.}
	\label{fig:rust_actix_web}
\end{figure}

Jak widać na Rysunku~\ref{fig:rust_actix_web}, Actix wypada bardzo korzystnie w kontekście wydajności i obsługi dużego ruchu, co jest istotne dla aplikacji wymagających niskich czasów odpowiedzi.

\section{Framework Tonic}

Tonic jest frameworkiem do implementacji gRPC w języku Rust, opartym na bibliotece \texttt{tokio} oraz protokole HTTP/2 \cite{tonic}. W projekcie został wykorzystany do:
\begin{itemize}[itemsep=0.2em, parsep=0pt, topsep=0pt]
	\item implementacji serwera gRPC,
	\item definiowania kontraktów komunikacyjnych w postaci plików \texttt{.proto},
	\item realizacji wydajnej komunikacji klient–serwer.
\end{itemize}

Framework Tonic umożliwił stworzenie stabilnej i szybkiej komunikacji typu RPC, co było kluczowe dla testów wydajnościowych i porównawczych z GraphQL.

\section{tungstenite-rs}

\texttt{tungstenite-rs} jest biblioteką w Rust umożliwiającą obsługę protokołu WebSocket. Została wykorzystana do \cite{tungstenite}:
\begin{itemize}[itemsep=0.2em, parsep=0pt, topsep=0pt]
	\item obsługi komunikacji w czasie rzeczywistym,
	\item przesyłania danych bez konieczności inicjowania nowych połączeń HTTP,
	\item testów alternatywnych modeli komunikacji.
\end{itemize}

\section{h2load}

\texttt{h2load} jest narzędziem służącym do testowania wydajności aplikacji korzystających z protokołu HTTP/2 \cite{h2load}. W projekcie zostało wykorzystane do:
\begin{itemize}[itemsep=0.2em, parsep=0pt, topsep=0pt]
	\item generowania dużej liczby równoległych zapytań,
	\item pomiaru czasów odpowiedzi serwera,
	\item analizy zachowania systemu pod obciążeniem.
\end{itemize}

\section{Python}

Python jest językiem wysokiego poziomu, który w projekcie został użyty głównie pomocniczo, do:
\begin{itemize}[itemsep=0.2em, parsep=0pt, topsep=0pt]
	\item analizy wyników testów wydajnościowych,
	\item generowania wykresów porównawczych,
	\item przetwarzania danych pomiarowych.
\end{itemize}

Python pozostaje jednym z najpopularniejszych języków programowania \cite{python}, co obrazuje Rysunek~\ref{fig:python_tiobe}.

\begin{figure}[h]
	\centering
	\includegraphics[width=0.7\textwidth]{images/python_tiobe_index}
	\caption{Pozycja języka Python w rankingu Tiobe Index.}
	\label{fig:python_tiobe}
\end{figure}

\section{GraphQL}

GraphQL jest językiem zapytań do API, pozwalającym klientowi precyzyjnie określić, jakie dane są potrzebne \cite{gql}. W projekcie został wykorzystany do:
\begin{itemize}[itemsep=0.2em, parsep=0pt, topsep=0pt]
	\item realizacji zapytań o złożone struktury danych,
	\item testów wydajnościowych i porównawczych.
\end{itemize}

\section{gRPC}

gRPC jest mechanizmem komunikacji RPC opartym na protokole HTTP/2 oraz formacie binarnym Protocol Buffers. W projekcie zostało wykorzystane do:
\begin{itemize}[itemsep=0.2em, parsep=0pt, topsep=0pt]
	\item komunikacji między komponentami systemu,
	\item przeprowadzania testów wydajnościowych,
\end{itemize}

\section{WebAssembly}

WebAssembly (Wasm) jest binarnym formatem wykonywalnym, pozwalającym uruchamiać kod napisany w Rust bezpośrednio w przeglądarce. W projekcie zostało wykorzystane do:
\begin{itemize}[itemsep=0.2em, parsep=0pt, topsep=0pt]
	\item uruchamiania części logiki aplikacji po stronie klienta,
	\item eksperymentalnego porównania wydajności z klasycznym JavaScriptem,
	\item zwiększenia wydajności aplikacji webowej.
\end{itemize}
\section{rumqttc}

\texttt{rumqttc} jest asynchroniczną biblioteką w języku Rust służącą do obsługi protokołu MQTT (Message Queuing Telemetry Transport). Umożliwia ona implementację klientów MQTT, oferując niskie opóźnienia oraz wysoką wydajność komunikacji \cite{rumqttc}.

W projekcie biblioteka \texttt{rumqttc} została wykorzystana do:
\begin{itemize}[itemsep=0.2em, parsep=0pt, topsep=0pt]
	\item testów komunikacji typu publish–subscribe,
	\item analizy wydajności lekkich protokołów komunikacyjnych,
	\item porównania MQTT z innymi mechanizmami wymiany danych.
\end{itemize}

\section{wasm-pack}

\texttt{wasm-pack} jest narzędziem wspierającym proces kompilacji projektów napisanych w języku Rust do formatu WebAssembly. Automatyzuje on generowanie pakietów WASM, integrację z ekosystemem JavaScript oraz publikację modułów.

W projekcie \texttt{wasm-pack} został wykorzystany do:
\begin{itemize}[itemsep=0.2em, parsep=0pt, topsep=0pt]
	\item kompilacji modułów Rust do WebAssembly,
	\item przygotowania interfejsów JavaScript do komunikacji z WASM,
	\item budowy aplikacji demonstracyjnych uruchamianych w przeglądarce.
\end{itemize}

\section{Juniper}

\texttt{Juniper} jest biblioteką w języku Rust służącą do implementacji serwerów GraphQL. Umożliwia definiowanie schematów, zapytań oraz mutacji w sposób silnie typowany, zgodny z paradygmatami języka Rust \cite{juniper}.

W projekcie biblioteka \texttt{Juniper} została użyta do:
\begin{itemize}[itemsep=0.2em, parsep=0pt, topsep=0pt]
	\item implementacji API GraphQL,
	\item realizacji zapytań o złożone struktury danych,
	\item testów wydajnościowych komunikacji GraphQL.
\end{itemize}

\section{Apache Avro}

Apache Avro jest binarnym formatem serializacji danych opartym na schematach, wykorzystywanym głównie w systemach przetwarzania danych i architekturach rozproszonych. Zapewnia kompaktową reprezentację danych oraz możliwość ewolucji schematów \cite{avro}.

W projekcie Apache Avro został wykorzystany do:
\begin{itemize}[itemsep=0.2em, parsep=0pt, topsep=0pt]
	\item testów wydajności serializacji i deserializacji danych,
	\item porównania z innymi formatami binarnymi i tekstowymi,
	\item analizy rozmiaru wynikowych struktur danych.
\end{itemize}

\section{BSON}

BSON (Binary JSON) jest binarną reprezentacją formatu JSON, zaprojektowaną z myślą o efektywnej serializacji danych. Jest powszechnie stosowany m.in. w bazach danych MongoDB \cite{bson}.

W projekcie format BSON został wykorzystany do:
\begin{itemize}[itemsep=0.2em, parsep=0pt, topsep=0pt]
	\item testów wydajności serializacji danych,
	\item porównania binarnych i tekstowych formatów wymiany danych,
	\item analizy narzutu pamięciowego.
\end{itemize}

\section{Serde}

\texttt{Serde} jest biblioteką w języku Rust służącą do serializacji i deserializacji struktur danych \cite{serde}. Zapewnia zunifikowany interfejs obsługujący wiele formatów, takich jak JSON, BSON, XML czy MessagePack.

W projekcie \texttt{Serde} została wykorzystana jako:
\begin{itemize}[itemsep=0.2em, parsep=0pt, topsep=0pt]
	\item warstwa abstrakcji dla procesów serializacji danych,
	\item wspólna baza dla testów różnych formatów danych,
	\item narzędzie upraszczające implementację testów porównawczych.
\end{itemize}

\section{Prost}

\texttt{Prost} jest biblioteką w języku Rust służącą do obsługi formatu Protocol Buffers. Umożliwia generowanie kodu Rust na podstawie plików \texttt{.proto} oraz efektywną serializację danych binarnych \cite{prost}.

W projekcie \texttt{Prost} została wykorzystana do:
\begin{itemize}[itemsep=0.2em, parsep=0pt, topsep=0pt]
	\item obsługi komunikacji gRPC,
	\item serializacji danych w formacie Protocol Buffers,
	\item testów wydajności komunikacji RPC.
\end{itemize}

\section{quick-xml}

\texttt{quick-xml} jest szybką biblioteką w języku Rust przeznaczoną do parsowania i generowania dokumentów XML. Charakteryzuje się niskim narzutem pamięciowym oraz wysoką wydajnością \cite{qXML}.

W projekcie biblioteka \texttt{quick-xml} została użyta do:
\begin{itemize}[itemsep=0.2em, parsep=0pt, topsep=0pt]
	\item testów serializacji i deserializacji danych w formacie XML,
	\item porównania XML z innymi formatami wymiany danych,
	\item analizy wydajności przetwarzania danych tekstowych.
\end{itemize}
\chapter{Użyte narzędzia programistyczne}

\section{Git}
Git jest rozproszonym systemem kontroli wersji, którego głównym celem jest zarządzanie historią zmian w kodzie źródłowym projektu programistycznego \cite{git}. Narzędzie to umożliwia rejestrowanie kolejnych wersji plików, analizę wprowadzonych modyfikacji oraz powrót do wcześniejszych stanów projektu. Dzięki rozproszonej architekturze każdy użytkownik posiada pełną kopię repozytorium wraz z całą historią zmian, co zwiększa niezawodność oraz elastyczność pracy zespołowej.

Git oferuje mechanizmy takie jak gałęzie (ang. \textit{branches}) oraz scalanie zmian (ang. \textit{merge}), które pozwalają na równoległą pracę nad różnymi funkcjonalnościami bez ingerencji w główną wersję projektu. System ten jest szeroko stosowany zarówno w małych projektach indywidualnych, jak i w dużych przedsięwzięciach komercyjnych oraz otwartoźródłowych.

\section{Zed}
Zed jest nowoczesnym edytorem kodu źródłowego zaprojektowanym z myślą o wysokiej wydajności, niskich opóźnieniach oraz współczesnych potrzebach programistów \cite{zed}. Narzędzie to zostało stworzone przy wykorzystaniu języka Rust, co przekłada się na bezpieczeństwo pamięci oraz wysoką responsywność interfejsu użytkownika.

Edytor oferuje wsparcie dla wielu języków programowania, inteligentne podpowiedzi kodu, integrację z systemami kontroli wersji oraz możliwość pracy zespołowej w czasie rzeczywistym. Zed kładzie duży nacisk na minimalizm interfejsu oraz płynność pracy, co czyni go atrakcyjną alternatywą dla bardziej rozbudowanych środowisk programistycznych.

\section{Visual Studio Code}
Visual Studio Code jest wieloplatformowym edytorem kodu źródłowego rozwijanym przez firmę Microsoft \cite{vscode}. Narzędzie to łączy w sobie prostotę edytora tekstu z funkcjonalnością zintegrowanego środowiska programistycznego (IDE). Dzięki rozbudowanemu systemowi rozszerzeń możliwe jest dostosowanie edytora do pracy z niemal dowolnym językiem programowania oraz frameworkiem.

Visual Studio Code oferuje funkcje takie jak podświetlanie składni, inteligentne uzupełnianie kodu, debugowanie aplikacji oraz integrację z systemem Git. Jego popularność wynika z dużej elastyczności, aktywnej społeczności oraz regularnych aktualizacji, co czyni go jednym z najczęściej wybieranych narzędzi programistycznych.

\section{NPM}
NPM (Node Package Manager) jest menedżerem pakietów przeznaczonym dla środowiska Node.js \cite{npm}. Jego głównym zadaniem jest zarządzanie zależnościami projektu, w tym instalowanie, aktualizowanie oraz usuwanie bibliotek zewnętrznych. NPM umożliwia również publikowanie własnych pakietów w publicznym repozytorium, co wspiera ponowne wykorzystanie kodu.

Centralnym elementem działania NPM jest plik \texttt{package.json}, w którym definiowane są informacje o projekcie, jego zależnościach oraz skryptach automatyzujących typowe zadania, takie jak budowanie aplikacji czy uruchamianie testów. Narzędzie to stanowi podstawowy element ekosystemu JavaScript i jest powszechnie wykorzystywane w projektach frontendowych oraz backendowych.

\section{Cargo}
Cargo jest oficjalnym narzędziem do zarządzania projektami w języku Rust \cite{cargo}. Odpowiada ono za proces kompilacji kodu źródłowego, pobieranie i zarządzanie zależnościami oraz uruchamianie testów jednostkowych. Cargo integruje się bezpośrednio z kompilatorem Rust, zapewniając spójność i automatyzację procesu budowania aplikacji.

Konfiguracja projektu realizowana jest za pomocą pliku \texttt{Cargo.toml}, który zawiera informacje o projekcie, wersjach bibliotek oraz ustawieniach kompilacji. Dzięki Cargo proces tworzenia aplikacji w języku Rust jest uproszczony i ustandaryzowany, co sprzyja utrzymaniu wysokiej jakości kodu oraz jego skalowalności.

\section{Testy wydajnościowe}

W ramach przygotowanego środowiska testowego przeprowadzono szereg eksperymentów mających na celu obiektywną ocenę wydajności różnych mechanizmów komunikacji oraz formatów serializacji danych. Poniżej przedstawiono szczegółowy opis metodologii testowania poszczególnych grup aplikacji.

\subsection{Aplikacje zawarte w katalogu communication-mechanisms}

Dla każdego zaimplementowanego protokołu komunikacji przeprowadzono cztery kategorie testów wydajnościowych, pozwalających na kompleksową ocenę charakterystyk przesyłania danych:

\begin{enumerate}
	\item \textbf{Transmisja małych pakietów danych} -- test polegający na cyklicznym odpytywaniu serwera o dane o rozmiarze 1~KB. Pomiar ten pozwala ocenić narzut protokołu oraz czas odpowiedzi dla typowych żądań zawierających niewielką ilość informacji.
	
	\item \textbf{Transmisja dużych pakietów danych} -- test analogiczny do poprzedniego, z tą różnicą, że rozmiar przesyłanych danych wynosi 1~MB. Eksperyment ten umożliwia ocenę wydajności protokołów w scenariuszu transferu większych zasobów.
	
	\item \textbf{Head-of-Line Blocking} -- test weryfikujący występowanie zjawiska blokowania głowy kolejki, w którym opóźnienie w przetwarzaniu jednego żądania wpływa na obsługę kolejnych żądań w ramach tego samego połączenia. Pomiar ten jest szczególnie istotny dla protokołów wykorzystujących multipleksowanie strumieni.
	
	\item \textbf{Strumieniowanie danych} -- test przeprowadzany wyłącznie dla protokołów wspierających tryb strumieniowy (streaming). Ocenie podlega efektywność przesyłania danych w sposób ciągły, bez konieczności oczekiwania na kompletną odpowiedź serwera.
\end{enumerate}

\subsection{Aplikacje zawarte w katalogu data-serialization-formats}

Dla każdej z zaimplementowanych bibliotek serializacji danych przeprowadzono pomiary wydajności na podstawie ustandaryzowanej struktury danych. W celu zapewnienia porównywalności wyników, wszystkie testy wykorzystują identyczny zestaw danych testowych składający się z kolekcji 1000 obiektów użytkowników wraz z metadanymi.

Struktura danych testowych jest generowana w następujący sposób:

\begin{lstlisting}[style=rustcode, caption={Struktura danych wykorzystana w testach serializacji.}, label={lst:user_struct}]
	let count = 1000;
	let users: Vec<User> = (0..count)
	.map(|i| User {
		id: i as i64,
		name: format!("User {}", i),
		email: format!("user{}@example.com", i),
		age: 20 + (i % 50) as i32,
		is_active: i % 2 == 0,
		tags: vec![
		"tag1".to_string(),
		"tag2".to_string(),
		"tag3".to_string(),
		],
	})
	.collect();
	
	UserCollection {
		users,
		metadata: Metadata {
			version: "1.0.0".to_string(),
			created_at: "2025-01-23T00:00:00Z".to_string(),
			total_count: count,
		},
	}
\end{lstlisting}

Każdy obiekt użytkownika zawiera kompletny zestaw atrybutów: identyfikator liczbowy, nazwę, adres email, wiek, status aktywności oraz listę trzech tagów. Dodatkowo, kolekcja jest wzbogacona o metadane zawierające wersję struktury danych, znacznik czasu utworzenia oraz łączną liczbę elementów. Taka konstrukcja zapewnia reprezentatywny zestaw różnorodnych typów danych (liczby całkowite, ciągi znaków, wartości logiczne, kolekcje) występujących w rzeczywistych zastosowaniach.

Dla każdego formatu serializacji mierzono następujące parametry wydajnościowe:

\begin{enumerate}
	\item \textbf{Czas serializacji} -- czas potrzebny na przekształcenie struktury danych z reprezentacji obiektowej języka Rust do formatu serializowanego. Pomiar wyrażony w mikrosekundach ($\mu$s) i uśredniony na podstawie wielokrotnych iteracji.
	
	\item \textbf{Czas deserializacji} -- czas wymagany do odtworzenia struktury obiektowej z danych w formacie serializowanym. Pomiar wyrażony w mikrosekundach ($\mu$s) i uśredniony na podstawie wielokrotnych iteracji.
	
	\item \textbf{Rozmiar zserializowanych danych} -- całkowita wielkość danych po procesie serializacji, wyrażona w bajtach (B). Parametr ten pozwala ocenić efektywność kompresji oraz narzut protokołu dla różnych formatów wymiany danych.
\end{enumerate}

Wykorzystanie identycznego zestawu danych testowych dla wszystkich formatów serializacji umożliwia obiektywne porównanie ich wydajności oraz charakterystyk w zakresie rozmiaru wynikowych struktur danych.

\subsection{Testy przeglądarkowe}

Testy przeglądarkowe służą do porównania wydajności implementacji algorytmów w czystym języku JavaScript oraz w technologii WebAssembly. Eksperyment polega na równoległym uruchomieniu identycznych funkcjonalności w obu środowiskach wykonawczych i pomiarze czasu ich wykonania.

Testowane są następujące scenariusze:

\begin{itemize}
	\item \textbf{Implementacja JavaScript} -- natywny kod uruchamiany bezpośrednio przez silnik JavaScript przeglądarki,
	\item \textbf{Implementacja WebAssembly} -- kod napisany w języku Rust, skompilowany do formatu WASM za pomocą narzędzia \texttt{wasm-pack} i wykonywany w środowisku przeglądarki.
\end{itemize}

Porównanie obu implementacji pozwala na obiektywną ocenę potencjalnych korzyści wydajnościowych wynikających z zastosowania technologii WebAssembly w aplikacjach webowych wymagających intensywnych obliczeń.
\chapter{Przeprowadzanie testów}
Niniejszy rozdział prezentuje metodologię przeprowadzonych badań eksperymentalnych oraz ich wyniki ilościowe.

\section{Metodologia testowania protokołów komunikacyjnych}
Wszystkie eksperymenty zostały przeprowadzone z wykorzystaniem ujednoliconych parametrów testowych, dostosowanych do specyfiki każdego badanego protokołu:
\begin{itemize}
	\item Liczba współbieżnych operacji: 100
	\item Liczba zapytań na wątek: 1000
	\item Rozmiar standardowego ładunku danych: 1024 bajty (1 KB)
	\item Rozmiar rozszerzonego ładunku danych: 1048576 bajty (1 MB)
\end{itemize}

\section{Architektura REST API}
Implementacja architektury REST została zrealizowana przy użyciu biblioteki Actix-web -- jednej z najpopularniejszych wysokopoziomowych platform wspierających natywnie strumieniowanie danych oraz protokół HTTP/2 w połączeniu z certyfikatami TLS. W ramach analizy zbadano również problem blokowania na początku kolejki (Head-of-Line blocking, HOL) wraz z proponowanym rozwiązaniem.

\subsection{Wyniki dla protokołu HTTP/1.1}

\begin{figure}[H]
	\centering
	\includegraphics[width=0.55\textwidth]{images/communication-mechanisms/REST/http1/test-2026-02-01_14-11-27/req_per_sec.png}
	\caption{Przepustowość REST HTTP/1.1 dla ładunku 1 KB.}
	\label{fig:rest_http1_test}
\end{figure}

Przepustowość dla zapytań ze standardowym ładunkiem (1 KB) wyniosła 764076 żądań na sekundę.

\begin{figure}[H]
	\centering
	\includegraphics[width=0.55\textwidth]{images/communication-mechanisms/REST/http1/test_high_payload-2026-02-01_14-12-55/req_per_sec.png}
	\caption{Przepustowość REST HTTP/1.1 dla ładunku 1 MB.}
	\label{fig:rest_http1_test_high_payload}
\end{figure}

Dla rozszerzonego ładunku (1 MB) odnotowano przepustowość na poziomie 2759 żądań na sekundę, co stanowi spadek o 99,6\% w porównaniu ze standardowym ładunkiem.

\begin{figure}[H]
	\centering
	\includegraphics[width=0.55\textwidth]{images/communication-mechanisms/REST/http1/stream-2026-02-01_17-43-49/req_per_sec.png}
	\caption{Przepustowość REST HTTP/1.1 dla transmisji strumieniowej (100 iteracji).}
	\label{fig:rest_http1_stream}
\end{figure}

Transmisja strumieniowa składająca się ze 100 fragmentów (chunków) osiągnęła przepustowość 239056 żądań na sekundę.

\begin{figure}[H]
	\centering
	\includegraphics[width=0.55\textwidth]{images/communication-mechanisms/REST/http1/slow-2026-02-01_17-47-47/req_per_sec.png}
	\caption{Wpływ blokowania HOL na przepustowość REST HTTP/1.1.}
	\label{fig:rest_http1_hol_problem}
\end{figure}

W warunkach występowania problemu Head-of-Line blocking zmierzono przepustowość wynoszącą 1954 żądania na sekundę, demonstrując znaczącą degradację wydajności.

\subsection{Wyniki dla protokołu HTTP/2}

\begin{figure}[H]
	\centering
	\includegraphics[width=0.55\textwidth]{images/communication-mechanisms/REST/http2/test-2026-02-01_14-24-59/req_per_sec.png}
	\caption{Przepustowość REST HTTP/2 dla ładunku 1 KB.}
	\label{fig:rest_http2_test}
\end{figure}

Implementacja z wykorzystaniem HTTP/2 osiągnęła przepustowość 198272 żądań na sekundę dla standardowego ładunku -- wartość o 74\% niższą niż w przypadku HTTP/1.1.

\begin{figure}[H]
	\centering
	\includegraphics[width=0.55\textwidth]{images/communication-mechanisms/REST/http2/test_high_payload-2026-02-01_14-26-26/req_per_sec.png}
	\caption{Przepustowość REST HTTP/2 dla ładunku 1 MB.}
	\label{fig:rest_http2_test_high_payload}
\end{figure}

Dla dużych pakietów danych zaobserwowano przepustowość 1690 żądań na sekundę, co stanowi spadek o 39\% względem HTTP/1.1.

\begin{figure}[H]
	\centering
	\includegraphics[width=0.55\textwidth]{images/communication-mechanisms/REST/http2/stream-2026-02-01_17-45-13/req_per_sec.png}
	\caption{Przepustowość transmisji strumieniowej REST HTTP/2.}
	\label{fig:rest_http2_stream}
\end{figure}

Transmisja strumieniowa w protokole HTTP/2 uzyskała przepustowość 91687 żądań na sekundę, wykazując degradację o 62\% w stosunku do HTTP/1.1.

\begin{figure}[H]
	\centering
	\includegraphics[width=0.55\textwidth]{images/communication-mechanisms/REST/http2/slow-2026-02-01_17-45-46/req_per_sec.png}
	\caption{Mitigacja blokowania HOL w REST HTTP/2.}
	\label{fig:rest_http2_hol}
\end{figure}

W scenariuszu testowym HOL blocking protokół HTTP/2 osiągnął przepustowość 135372 żądań na sekundę, demonstrując 69-krotną poprawę względem HTTP/1.1.

\section{Architektura GraphQL}
Implementację GraphQL zrealizowano przy użyciu biblioteki Juniper -- zaawansowanego frameworka wspierającego natywnie strumieniowanie oraz HTTP/2 z wykorzystaniem TLS.

\subsection{Wyniki dla protokołu HTTP/1.1}

\begin{figure}[H]
	\centering
	\includegraphics[width=0.55\textwidth]{images/communication-mechanisms/graphql/http1/test-2026-02-01_14-37-46/req_per_sec.png}
	\caption{Przepustowość GraphQL HTTP/1.1 dla ładunku 1 KB.}
	\label{fig:gql_http1_test}
\end{figure}

GraphQL w połączeniu z HTTP/1.1 osiągnął przepustowość 179760 żądań na sekundę dla standardowego ładunku, co stanowi 76\% wydajności REST API.

\begin{figure}[H]
	\centering
	\includegraphics[width=0.55\textwidth]{images/communication-mechanisms/graphql/http1/large_payload-2026-02-01_14-39-00/req_per_sec.png}
	\caption{Przepustowość GraphQL HTTP/1.1 dla ładunku 1 MB.}
	\label{fig:gql_http1_test_high_payload}
\end{figure}

Dla rozszerzonego ładunku zmierzono przepustowość 2418 żądań na sekundę, wykazując nieznaczną przewagę (12\%) nad REST API.

\subsection{Wyniki dla protokołu HTTP/2}

\begin{figure}[H]
	\centering
	\includegraphics[width=0.55\textwidth]{images/communication-mechanisms/graphql/http2/test-2026-02-01_14-40-00/req_per_sec.png}
	\caption{Przepustowość GraphQL HTTP/2 dla ładunku 1 KB.}
	\label{fig:gql_http2_test}
\end{figure}

Konfiguracja z HTTP/2 osiągnęła przepustowość 195877 żądań na sekundę, demonstrując 9\% poprawę względem HTTP/1.1 oraz porównywalne wyniki z REST HTTP/2.

\begin{figure}[H]
	\centering
	\includegraphics[width=0.55\textwidth]{images/communication-mechanisms/graphql/http2/large_payload-2026-02-01_14-41-05/req_per_sec.png}
	\caption{Przepustowość GraphQL HTTP/2 dla ładunku 1 MB.}
	\label{fig:gql_http2_test_high_payload}
\end{figure}

Dla dużych pakietów danych zaobserwowano przepustowość 2097 żądań na sekundę, co stanowi 24\% wzrost względem REST HTTP/2.

\section{Architektura gRPC}
Implementacja gRPC została zrealizowana przy użyciu biblioteki Tonic -- wysoko wydajnego frameworka natywnie wspierającego HTTP/2, strumieniowanie danych oraz szyfrowanie TLS. Ze względu na optymalizację wydajności zrezygnowano z implementacji mechanizmu refleksji API.

\subsection{Wyniki dla protokołu HTTP/2}

\begin{figure}[H]
	\centering
	\includegraphics[width=0.55\textwidth]{images/communication-mechanisms/grpc/test-2026-02-01_13-21-44/req_per_sec.png}
	\caption{Przepustowość gRPC dla ładunku 1 KB.}
	\label{fig:grpc_http2_test}
\end{figure}

gRPC osiągnął przepustowość 168060 żądań na sekundę dla standardowego ładunku, co stanowi 15\% spadek względem GraphQL HTTP/2.

\begin{figure}[H]
	\centering
	\includegraphics[width=0.55\textwidth]{images/communication-mechanisms/grpc/large_payload-2026-02-01_14-43-53/req_per_sec.png}
	\caption{Przepustowość gRPC dla ładunku 1 MB.}
	\label{fig:grpc_http2_test_high_payload}
\end{figure}

Dla rozszerzonego ładunku zmierzono przepustowość 2243 żądań na sekundę, wykazując 7\% przewagę nad GraphQL oraz 33\% nad REST w konfiguracji HTTP/2.

\section{Architektura WebSocket}
Do implementacji architektury WebSocket wykorzystano bibliotekę tungstenite-rs -- popularny framework wysokiego poziomu dedykowany komunikacji dwukierunkowej w czasie rzeczywistym.

\subsection{Wyniki dla protokołu WebSocket}

\begin{figure}[H]
	\centering
	\includegraphics[width=0.55\textwidth]{images/communication-mechanisms/websocket/test-2026-02-01_16-55-46/req_per_sec.png}
	\caption{Przepustowość WebSocket dla ładunku 1 KB.}
	\label{fig:ws_test}
\end{figure}

Protokół WebSocket osiągnął przepustowość 122431 żądań na sekundę, wykazując najniższą wartość spośród testowanych protokołów dla standardowego ładunku.

\begin{figure}[H]
	\centering
	\includegraphics[width=0.55\textwidth]{images/communication-mechanisms/websocket/test_high_payload-2026-02-01_16-56-43/req_per_sec.png}
	\caption{Przepustowość WebSocket dla ładunku 1 MB.}
	\label{fig:ws_test_high_payload}
\end{figure}

Dla dużych pakietów danych zaobserwowano przepustowość 3187 żądań na sekundę -- najwyższą wartość wśród wszystkich badanych protokołów, przewyższającą o 42\% najbliższego konkurenta (gRPC).

\subsection{Wyniki dla protokołu WebSocket Secure}

\begin{figure}[H]
	\centering
	\includegraphics[width=0.55\textwidth]{images/communication-mechanisms/websocket_secure/test-2026-02-01_17-08-26/req_per_sec.png}
	\caption{Przepustowość WSS dla ładunku 1 KB.}
	\label{fig:wss_test}
\end{figure}

Wersja zabezpieczona (WSS) osiągnęła przepustowość 111086 żądań na sekundę, co stanowi 9\% degradację względem niezabezpieczonej implementacji.

\begin{figure}[H]
	\centering
	\includegraphics[width=0.55\textwidth]{images/communication-mechanisms/websocket_secure/test_high_payload-2026-02-01_17-10-14/req_per_sec.png}
	\caption{Przepustowość WSS dla ładunku 1 MB.}
	\label{fig:wss_test_high_payload}
\end{figure}

Dla rozszerzonego ładunku zmierzono przepustowość 2932 żądań na sekundę, wykazując 8\% spadek względem wersji niezabezpieczonej.

\section{Architektura SOAP}
Implementacja protokołu SOAP została przetestowana w obu wersjach HTTP w celu porównania charakterystyk wydajnościowych.

\subsection{Wyniki dla protokołu HTTP/1.1}

\begin{figure}[H]
	\centering
	\includegraphics[width=0.55\textwidth]{images/communication-mechanisms/SOAP/http1/test-2026-02-01_14-06-11/req_per_sec.png}
	\caption{Przepustowość SOAP HTTP/1.1 dla ładunku 1 KB.}
	\label{fig:soap_http1_test}
\end{figure}

SOAP z HTTP/1.1 osiągnął przepustowość 198232 żądań na sekundę, demonstrując porównywalne wyniki z REST HTTP/2.

\begin{figure}[H]
	\centering
	\includegraphics[width=0.55\textwidth]{images/communication-mechanisms/SOAP/http1/test_high_payload-2026-02-01_14-05-50/req_per_sec.png}
	\caption{Przepustowość SOAP HTTP/1.1 dla ładunku 1 MB.}
	\label{fig:soap_http1_high_payload}
\end{figure}

Dla rozszerzonego ładunku zmierzono przepustowość 2892 żądań na sekundę -- najwyższą wartość wśród protokołów wykorzystujących HTTP/1.1.

\subsection{Wyniki dla protokołu HTTP/2}

\begin{figure}[H]
	\centering
	\includegraphics[width=0.55\textwidth]{images/communication-mechanisms/SOAP/http2/test-2026-02-01_14-08-18/req_per_sec.png}
	\caption{Przepustowość SOAP HTTP/2 dla ładunku 1 KB.}
	\label{fig:soap_http2_test}
\end{figure}

Implementacja z HTTP/2 osiągnęła przepustowość 213973 żądań na sekundę, wykazując 8\% poprawę względem HTTP/1.1 oraz najwyższy wynik spośród wszystkich protokołów HTTP/2.

\begin{figure}[H]
	\centering
	\includegraphics[width=0.55\textwidth]{images/communication-mechanisms/SOAP/http2/large_payload_test-2026-02-01_14-09-26/req_per_sec.png}
	\caption{Przepustowość SOAP HTTP/2 dla ładunku 1 MB.}
	\label{fig:soap_http2_high_payload}
\end{figure}

Dla dużych pakietów danych zaobserwowano przepustowość 2478 żądań na sekundę, co stanowi 14\% spadek względem HTTP/1.1.

\section{Protokół MQTT}

\begin{figure}[H]
	\centering
	\includegraphics[width=0.55\textwidth]{images/communication-mechanisms/mqtt/test-2026-02-01_18-35-29/req_per_sec.png}
	\caption{Przepustowość MQTT dla ładunku 1 KB.}
	\label{fig:mqtt_test}
\end{figure}

Protokół MQTT osiągnął przepustowość 112126 żądań na sekundę dla standardowego ładunku.

\begin{figure}[H]
	\centering
	\includegraphics[width=0.55\textwidth]{images/communication-mechanisms/mqtt/high_payload-2026-02-01_18-34-45/req_per_sec.png}
	\caption{Przepustowość MQTT dla ładunku 1 MB.}
	\label{fig:mqtt_test_high_payload}
\end{figure}

Dla rozszerzonego ładunku zmierzono przepustowość 2167 żądań na sekundę, lokując MQTT w środkowym przedziale wydajnościowym.

\section{Analiza porównawcza formatów serializacji danych}

Przeprowadzono testy wydajnościowe dla sześciu popularnych formatów serializacji, mierząc czas serializacji, deserializacji oraz rozmiar wynikowych danych. Testowy zbiór składał się z 1000 obiektów użytkownika wraz z metadanymi.

\subsection{Apache Avro}

\begin{figure}[H]
	\centering
	\includegraphics[width=0.55\textwidth]{images/data-serializations/avro-2026-01-29_15-45-46/serialization_metrics.png}
	\caption{Metryki wydajności Apache Avro.}
	\label{fig:avro}
\end{figure}

Apache Avro uzyskał czas serializacji 7,396 ms, deserializacji 5,988 ms przy rozmiarze danych wynoszącym 521727 bajtów.

\subsection{Binary JSON}

\begin{figure}[H]
	\centering
	\includegraphics[width=0.55\textwidth]{images/data-serializations/bson-2026-01-29_15-48-49/serialization_metrics.png}
	\caption{Metryki wydajności Binary JSON.}
	\label{fig:bson}
\end{figure}

Format BSON wykazał czas serializacji 2,073 ms, deserializacji 4,842 ms, generując dane o rozmiarze 1426779 bajtów -- największym spośród testowanych formatów.

\subsection{JSON}

\begin{figure}[H]
	\centering
	\includegraphics[width=0.55\textwidth]{images/data-serializations/json-2026-01-29_16-15-15/serialization_metrics.png}
	\caption{Metryki wydajności JSON.}
	\label{fig:json}
\end{figure}

Standardowy JSON osiągnął czas serializacji 1,294 ms oraz deserializacji 3,565 ms przy rozmiarze 1181768 bajtów, oferując najszybszą deserializację wśród formatów tekstowych.

\subsection{MessagePack}

\begin{figure}[H]
	\centering
	\includegraphics[width=0.55\textwidth]{images/data-serializations/message_pack-2026-01-29_16-21-25/serialization_metrics.png}
	\caption{Metryki wydajności MessagePack.}
	\label{fig:message-pack}
\end{figure}

MessagePack wykazał najlepszą wydajność czasową z serializacją 0,748 ms i deserializacją 2,247 ms, przy kompaktowym rozmiarze 527431 bajtów.

\subsection{Protocol Buffers}

\begin{figure}[H]
	\centering
	\includegraphics[width=0.55\textwidth]{images/data-serializations/protobuf-2026-01-29_16-23-15/serialization_metrics.png}
	\caption{Metryki wydajności Protocol Buffers.}
	\label{fig:protobuf}
\end{figure}

Protocol Buffers osiągnął czas serializacji 1,648 ms, deserializacji 4,724 ms, generując dane o rozmiarze 587684 bajtów.

\subsection{XML}

\begin{figure}[H]
	\centering
	\includegraphics[width=0.55\textwidth]{images/data-serializations/xml-2026-01-29_16-24-52/serialization_metrics.png}
	\caption{Metryki wydajności XML.}
	\label{fig:xml}
\end{figure}

Format XML wykazał najdłuższe czasy przetwarzania -- serializację 5,273 ms i deserializację 18,078 ms -- przy największym rozmiarze danych 1761825 bajtów, co stanowi 3,4-krotność najbardziej efektywnych formatów binarnych.

\section{Analiza porównawcza protokołów komunikacyjnych}

Przeprowadzone badania umożliwiły kompleksową ocenę wydajności protokołów komunikacyjnych w różnych scenariuszach użycia. Poniższa analiza syntetyzuje kluczowe wnioski z testów.

\subsection{Wpływ rozmiaru ładunku na wydajność}

\begin{table}[H]
	\centering
	\caption{Porównanie przepustowości protokołów komunikacyjnych}
	\label{tab:protocol_comparison}
	\begin{tabular}{|l|r|r|r|}
		\hline
		\textbf{Protokół} & \textbf{1 KB [req/s]} & \textbf{1 MB [req/s]} & \textbf{Degradacja} \\
		\hline
		REST HTTP/1.1 & 764,076 & 2,759 & 276.9x \\
		REST HTTP/2 & 198,272 & 1,690 & 117.3x \\
		GraphQL HTTP/1.1 & 179,760 & 2,418 & 74.3x \\
		GraphQL HTTP/2 & 195,877 & 2,097 & 93.4x \\
		gRPC & 168,060 & 2,243 & 74.9x \\
		WebSocket & 122,431 & 3,187 & 38.4x \\
		WebSocket Secure & 111,086 & 2,932 & 37.9x \\
		SOAP HTTP/1.1 & 198,232 & 2,892 & 68.5x \\
		SOAP HTTP/2 & 213,973 & 2,478 & 86.3x \\
		MQTT & 112,126 & 2,167 & 51.7x \\
		\hline
	\end{tabular}
\end{table}

Analiza współczynnika degradacji (tabela \ref{tab:protocol_comparison}) ujawnia fundamentalne różnice w charakterystykach wydajnościowych protokołów. WebSocket Secure wykazał najniższą degradację (37.9x), co potwierdza jego optymalizację dla transmisji dużych wolumenów danych poprzez utrzymywanie stałego połączenia. W przeciwieństwie do tego, REST HTTP/1.1 odnotował najwyższą degradację (276.9x), co jest konsekwencją narzutu związanego z nawiązywaniem nowego połączenia TCP dla każdego żądania oraz brakiem multipleksowania.

\begin{figure}[H]
	\centering
	\includegraphics[width=0.85\textwidth]{images/protocol_comparison_grouped.png}
	\caption{Porównanie przepustowości protokołów dla ładunku 1 KB i 1 MB (skala logarytmiczna).}
	\label{fig:protocol_comparison_grouped}
\end{figure}

Wykres \ref{fig:protocol_comparison_grouped} ilustruje wyraźną dychotomię między wydajnością dla małych i dużych ładunków. Protokoły oparte na HTTP/1.1 dominują w scenariuszach małych pakietów, podczas gdy protokoły z trwałymi połączeniami (WebSocket, WSS) wykazują przewagę dla dużych transferów danych.

\subsection{Wpływ TLS na wydajność}

Bezpieczeństwo transmisji za pomocą TLS wprowadza dodatkowy narzut wydajnościowy, jednak jest niezbędne w aplikacjach produkcyjnych. Kluczowe aspekty TLS handshake:

\begin{itemize}
	\item \textbf{Czas inicjalizacji}: TLS handshake typowo dodaje 50-100 ms opóźnienia dla pierwszego połączenia
	\item \textbf{Narzut obliczeniowy}: Szyfrowanie symetryczne (AES) ma minimalny wpływ na przepustowość (<5\%)
	\item \textbf{Optymalizacja}: HTTP/2 i WebSocket amortyzują koszt TLS poprzez długotrwałe połączenia
	\item \textbf{Session resumption}: TLS 1.3 redukuje czas ponownego handshake'a do ~1 RTT
\end{itemize}

Porównanie WebSocket (122,431 req/s) vs WebSocket Secure (111,086 req/s) pokazuje 9\% degradację -- znacząco niższą niż REST HTTP/1.1 vs HTTP/2 (74\%), co wskazuje, że jednorazowy koszt TLS handshake jest amortyzowany w długotrwałych połączeniach.

\begin{figure}[H]
	\centering
	\includegraphics[width=0.85\textwidth]{images/protocol_degradation.png}
	\caption{Współczynnik degradacji wydajności przy zwiększeniu ładunku z 1 KB do 1 MB.}
	\label{fig:protocol_degradation}
\end{figure}

\subsection{Rekomendacje wyboru protokołu}

Na podstawie przeprowadzonych badań, dobór protokołu powinien uwzględniać specyficzne wymagania aplikacji:

\begin{enumerate}
	\item \textbf{REST API (HTTP/1.1 lub HTTP/2)}
	\begin{itemize}
		\item Zastosowanie: Aplikacje webowe, API publiczne, architektury mikroserwisowe
		\item Zalety: Najlepsza skalowalność, cache'owalność, powszechne wsparcie
		\item Wybór HTTP/2: Gdy priorytetem jest ograniczenie head of line blocking i bezpieczeństwo
	\end{itemize}
	
	\item \textbf{GraphQL}
	\begin{itemize}
		\item Zastosowanie: aplikacje frontendowe, złożone modele danych
		\item Zalety: Eliminacja over-fetchingu, redukcja liczby zapytań
		\item Kompromis: ~1-2\% niższa wydajność niż REST przy znaczącej poprawie pisania oprogramowania
	\end{itemize}
	
	\item \textbf{gRPC}
	\begin{itemize}
		\item Zastosowanie: Komunikacja pomiędzy serwisami, wysoko wydajne współczesne rozwiązania backendowe
		\item Zalety: Najlepsza wydajność dla dużych ładunków w HTTP/2, silne kontrakty, eliminacja problemu pomiędzy językami programowania.
		\item Ograniczenie: Słabsze wsparcie w przeglądarkach (wymaga gRPC-web)
	\end{itemize}
	
	\item \textbf{WebSocket/WSS}
	\begin{itemize}
		\item Zastosowanie: Aplikacje czasu rzeczywistego (czaty, giełdy, gry)
		\item Zalety: Najniższa latencja, najlepsza wydajność dla dużych ładunków
		\item Kompromis: Zwiększona złożoność skalowania. 
	\end{itemize}
	
	\item \textbf{SOAP}
	\begin{itemize}
		\item Zastosowanie: Systemy enterprise, integracje legacy, transakcje finansowe
		\item Zalety: Formalne standardy (WS-Security, WS-Transaction), compliance
		\item Ograniczenie: Narzut XML, większa złożoność implementacji
	\end{itemize}
	
	\item \textbf{MQTT}
	\begin{itemize}
		\item Zastosowanie: IoT, urządzenia wbudowane, sieci z ograniczonym pasmem
		\item Zalety: Minimalny narzut protokołu, automatyczne kolejkowanie.
		\item Optymalne: Dla rozproszonych sieci.
	\end{itemize}
\end{enumerate}

\subsection{Kluczowe wnioski}

Przeprowadzone badania prowadzą do następujących konkluzji:

\begin{enumerate}
	\item \textbf{Brak uniwersalnego lidera}: Każdy protokół wykazuje przewagi w specyficznych scenariuszach użycia. REST HTTP/1.1 dominuje dla małych zapytań, WebSocket dla dużych transferów, gRPC dla komunikacji pomiędzy serwisami lub aplikacjami backendowymi.
	
	\item \textbf{TLS jest akceptowalnym kosztem}: Degradacja wydajności 5-15\% (w zależności od protokołu) jest w pełni uzasadniona przez krytyczne wymagania bezpieczeństwa. Nowoczesne implementacje TLS 1.3 dodatkowo minimalizują ten narzut.
	
	\item \textbf{Długotrwałe połączenia amortyzują koszty}: Protokoły takie jak WebSocket, HTTP/2 wykazują niższą degradację przy wzroście ładunku, co czyni je preferowanymi dla ładunków o wysokiej przepustowości.
	
	\item \textbf{Multipleksowanie rozwiązuje HOL blocking}: HTTP/2 osiągnęło 69-krotną poprawę w testach HOL, potwierdzając efektywność multipleksowania strumieni.
	
	\item \textbf{Specjalizacja protokołów IoT}: MQTT, mimo umiarkowanej przepustowości, pozostaje optymalnym wyborem dla IoT dzięki mechanizmom QoS i niskiej konsumpcji zasobów.
\end{enumerate}

\section{Analiza porównawcza WebAssembly i JavaScript}

\subsection{WebWorker}

\subsection{Canvas}

\chapter*{Zakończenie}
\addcontentsline{toc}{chapter}{Zakończenie}

Niniejsza praca magisterska poświęcona była kompleksowej analizie wydajności protokołów komunikacyjnych, formatów serializacji danych oraz technologii wykonawczych stosowanych we współczesnych aplikacjach webowych. Przeprowadzone badania eksperymentalne dostarczyły cennych danych ilościowych oraz pozwoliły na sformułowanie praktycznych rekomendacji dla architektów systemów i programistów.

\section*{Podsumowanie wyników badań}

Analiza protokołów komunikacyjnych wykazała, że nie istnieje uniwersalne rozwiązanie optymalne dla wszystkich scenariuszy użycia. REST API z protokołem HTTP/1.1 osiągnął najwyższą przepustowość dla małych pakietów danych (764~076 żądań na sekundę), co czyni go doskonałym wyborem dla mikrousług operujących na niewielkich zbiorach danych niewymagających bezpiecznego połączenia. Z kolei WebSocket wykazał najlepszą wydajność dla transferu dużych ładunków (3~187 żądań na sekundę) oraz najniższy współczynnik degradacji (38,4×), potwierdzając przewagę protokołów z trwałymi połączeniami w scenariuszach wymagających intensywnej wymiany danych.

Szczególnie istotnym odkryciem była efektywność protokołu HTTP/2 w mitigacji problemu Head-of-Line blocking. Osiągnięta 69-krotna poprawa przepustowości w porównaniu z HTTP/1.1 (135~372 vs. 1~954 żądań na sekundę) potwierdza, że mechanizm multipleksowania strumieni stanowi fundamentalną optymalizację dla nowoczesnych aplikacji webowych. Dodatkowym aspektem wartym podkreślenia jest stosunkowo niewielki narzut wydajnościowy związany z warstwą TLS -- degradacja rzędu 5--15\% jest w pełni akceptowalna w kontekście krytycznych wymagań bezpieczeństwa współczesnych systemów.

W zakresie formatów serializacji MessagePack osiągnął najkrótszy całkowity czas przetwarzania (2,995~ms), podczas gdy Apache Avro zapewnił najlepszą kompresję danych (509,5~KB). Format JSON, mimo 2,2-krotnie większego rozmiaru względem formatów binarnych, pozostaje wartościowym wyborem ze względu na czytelność, debugowalność oraz natywne wsparcie w przeglądarkach. Z kolei XML, choć wykazał najgorsze metryki wydajnościowe (23,4~ms łącznie, 1721~KB), zachowuje znaczenie w środowiskach enterprise ze względu na wymagania regulacyjne oraz wsparcie dla złożonych schematów walidacji.

Analiza technologii WebAssembly dostarczyła niuansowanego obrazu jej zastosowania w aplikacjach webowych. Kluczowym wnioskiem jest, że złożoność obliczeniowa stanowi czynnik decydujący o opłacalności implementacji w WebAssembly. Podczas gdy proste operacje (grayscale, brightness, contrast) wykazały lepszą wydajność w JavaScript, złożone algorytmy -- reprezentowane przez operację blur -- osiągnęły 3,52-krotne przyspieszenie w WebAssembly (371,3~ms vs. 1308,5~ms). Ten wzorzec potwierdza, że WebAssembly znajduje optymalne zastosowanie w scenariuszach wymagających intensywnych obliczeń numerycznych, podczas gdy JavaScript pozostaje preferowany dla operacji o niskiej złożoności oraz częstej interakcji z DOM.

\section*{Wkład naukowy i praktyczny}

Wartość niniejszej pracy przejawia się w kilku kluczowych aspektach. Po~pierwsze, przeprowadzone badania opierają się na rzeczywistych implementacjach oraz kontrolowanych eksperymentach, eliminując abstrakcyjne założenia teoretyczne. Ujednolicona metodologia testowania (100 współbieżnych operacji, 1000 zapytań na wątek) zapewnia porównywalność wyników między różnymi protokołami i formatami.

Po drugie, praca dostarcza konkretnych, kwantyfikowalnych metryk wydajnościowych, które mogą stanowić bezpośrednie wsparcie w procesie podejmowania decyzji architektonicznych. Zamiast ogólnych stwierdzeń typu ``protokół X jest szybszy od Y'', prezentowane są precyzyjne dane numeryczne wraz z analizą kontekstów, w których dana technologia wykazuje przewagę.

Po trzecie, sformułowane rekomendacje uwzględniają wielowymiarowy charakter wydajności systemu -- nie ograniczają się wyłącznie do przepustowości, ale uwzględniają również aspekty takie jak skalowalność, kompatybilność wsteczna, developer experience oraz wymagania bezpieczeństwa i compliance.

\section*{Ograniczenia badań}

Przy interpretacji wyników należy uwzględnić pewne ograniczenia przeprowadzonych badań. Testy wykonano w kontrolowanym środowisku testowym, które nie w pełni odzwierciedla złożoność środowisk produkcyjnych, charakteryzujących się zmiennym obciążeniem sieciowym, różnorodną infrastrukturą oraz nieprzewidywalnymi wzorcami użytkowania.

Dodatkowo, implementacje testowe zostały zoptymalizowane pod kątem maksymalnej wydajności dla poszczególnych protokołów, co może nie odzwierciedlać typowych wdrożeń produkcyjnych, w których priorytetem jest często równowaga między wydajnością a możliwością utrzymania kodu oraz zgodnością z istniejącą architekturą systemu.

Warto również zaznaczyć, że wyniki dotyczące WebAssembly odnoszą się do konkretnych implementacji algorytmów i mogą różnić się w zależności od charakterystyki przetwarzanych danych oraz specyfiki wykorzystywanych bibliotek.

\section*{Kierunki dalszych badań}

Przeprowadzone badania otwierają kilka perspektywicznych kierunków dalszych eksploracji. Pierwszym z nich jest analiza wydajności protokołów w środowiskach rozproszonych charakteryzujących się wysokim opóźnieniem sieciowym oraz niestabilnymi połączeniami, co jest szczególnie istotne w kontekście aplikacji mobilnych oraz systemów IoT.

Interesującym kierunkiem jest również badanie wpływu nowych wersji protokołów -- w szczególności HTTP/3 z protokołem QUIC -- na wydajność komunikacji w aplikacjach webowych. HTTP/3, eliminujący problem Head-of-Line blocking na poziomie warstwy transportowej, może istotnie zmienić krajobraz wydajnościowy prezentowany w niniejszej pracy.

Wreszcie, w kontekście rosnącej popularności architektur serverless oraz edge computing, wartościowe byłoby zbadanie, jak charakterystyki wydajnościowe poszczególnych protokołów i formatów zmieniają się w środowiskach o dynamicznej skalowalności i geograficznie rozproszonych węzłach obliczeniowych.

\section*{Wnioski końcowe}

Głównym przesłaniem niniejszej pracy jest przekonanie, że świadome i oparte na danych empirycznych decyzje architektoniczne stanowią fundament wydajnych systemów informatycznych. Wybór protokołu komunikacyjnego, formatu serializacji oraz technologii wykonawczej powinien być determinowany przez specyficzne wymagania aplikacji, uwzględniając takie czynniki jak charakterystyka przetwarzanych danych, oczekiwana latencja, wymagania bezpieczeństwa oraz ograniczenia infrastrukturalne.

Przeprowadzone badania potwierdzają, że nowoczesne aplikacje webowe mogą osiągać wysoką wydajność poprzez inteligentne wykorzystanie dostępnych technologii. REST API pozostaje doskonałym wyborem dla większości aplikacji webowych dzięki skalowalności i uniwersalności, WebSocket okazuje się optymalny dla komunikacji w czasie rzeczywistym, gRPC przewyższa konkurencję w komunikacji backend-to-backend, a WebAssembly stanowi wartościowe uzupełnienie JavaScript w scenariuszach wymagających intensywnych obliczeń.

Równocześnie praca podkreśla, że wydajność nie powinna być traktowana jako wyłączny czynnik decyzyjny -- aspekty takie jak możliwość utrzymania kodu, ekosystem narzędzi, wsparcie społeczności oraz kompatybilność z istniejącymi systemami często przeważają nad przewagą wydajnościową rzędu kilku czy kilkunastu procent.

Podsumowując, niniejsza praca dostarcza praktycznych wskazówek oraz danych empirycznych, które mogą wspierać proces projektowania i optymalizacji nowoczesnych aplikacji webowych. Mam nadzieję, że przedstawione wyniki i rekomendacje okażą się wartościowe dla architektów systemów, programistów oraz badaczy zajmujących się wydajnością aplikacji internetowych.


\bibliographystyle{plain}
\bibliography{literatura}


% spis rysunków
\listoffigures
% spis listingów 
\lstlistoflistings


% załączniki (opcjonalnie):
\appendix
%\chapter{Tytuł załącznika jeden}
%Treść załącznika jeden.

%\chapter{Tytuł załącznika dwa}
%Treść załącznika dwa.

%\bibliographystyle{plain}
%\bibliography{literatura}

%\printbibliography[type=book,title={Bibliografia - książki}]
%\printbibliography[type=misc,title={Bibliografia - strony internetowe}]

% spis tabel (jeżeli jest potrzebny):
%\listoftables

% spis rysunków (jeżeli jest potrzebny):
%\listoffigures

% spis listingów (jeżeli jest potrzebny):
%\lstlistoflistings

\end{document}
