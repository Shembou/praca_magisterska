%\documentclass[xodstep,a4paper,twoside,openany,12pt,openright]{wnspt}
\documentclass[a4paper,twoside,12pt]{report}
\usepackage[utf8]{inputenc}
\usepackage[T1]{fontenc}
\usepackage[lmargin=2cm,rmargin=2cm, tmargin=2.5cm,bmargin=2.5cm]{geometry}
\usepackage{amsmath}
\usepackage{amsfonts}
\usepackage{amsthm}
\usepackage{amssymb}
\usepackage{polski}
\usepackage{colortbl}
\usepackage{url}
\usepackage{setspace}
\usepackage{indentfirst}
\usepackage{listingsutf8}
\usepackage{beramono}
\usepackage{graphicx}
\usepackage{hyperref}
\usepackage{enumitem}

% --- opcje ---
\lstset{
	language=Java,
	basicstyle=\ttfamily,
	numbers=left,
	numberstyle=\footnotesize,
	stepnumber=1,
	numbersep=5pt,
	showspaces=false,
	showstringspaces=false,
	showtabs=false,
	frame=single,
	tabsize=4,
	captionpos=b,
	breaklines=true,
	escapeinside={\%*}{*)}
}

\newtheorem{lemat}{Lemat}
\newtheorem{twierdzenie}{Twierdzenie}

\renewcommand{\lstlistingname}{Listing}
\renewcommand{\lstlistlistingname}{Spis listingów}

% --- autor ---

\title{Sposoby optymalizacji aplikacji internetowych.

Ways to optimize web applications}


\begin{document}
\thispagestyle{empty}
\begin{center}
UNIWERSYTET JANA DŁUGOSZA W CZĘSTOCHOWIE 
\vspace{2em}\\
\includegraphics[width=0.2\textwidth]{logo_ujd.png}\\
\vspace{2em}
Wydział Nauk Ścisłych, Przyrodniczych i Technicznych
\end{center}

\vspace{2em}
\begin{flushleft}
Kierunek: \textbf{Informatyka}\\
Specjalność: \textbf{Programowanie gier komputerowych}
\end{flushleft}

\vspace{2em}
\begin{center}
\textbf{ \large Przemysław Kamiński}

Nr albumu: \textbf{88751}

\vspace{2em}

{\bfseries \Large 
Sposoby optymalizacji aplikacji internetowych. 

Ways to optimize web applications
}
\end{center}
\vfill
\vspace{2em}
\begin{flushright}
\begin{tabular}{l}
 Praca magisterska \\
 przygotowana pod kierunkiem\\
 dr hab. Bożeny Woźnej-Szcześniak, prof. UJD
\end{tabular}
\end{flushright}
\vfill
\vspace{2em}
\begin{center}
Częstochowa, 2025   
\end{center}

\newpage
\thispagestyle{empty}
\null\newpage
\begin{abstract}
	Celem niniejszej pracy jest porównanie wydajności protokołów komunikacyjnych stosowanych w aplikacjach webowych oraz analiza wpływu formatów wymiany danych na efektywność komunikacji w nowoczesnych systemach informatycznych. Badania obejmują protokoły HTTP/1.1, HTTP/2, GraphQL, gRPC, SOAP, WebSocket Secure oraz WebSocket, a także formaty danych takie jak JSON, XML, Protocol Buffers i MessagePack. Analiza koncentruje się na porównaniu rozmiaru przesyłanych danych, czasu przetwarzania oraz kompatybilności pomiędzy systemami.
	
	Dodatkowo w pracy oceniono wpływ zastosowania technologii WebAssembly na wydajność przetwarzania danych po stronie klienta w porównaniu z tradycyjnym podejściem opartym na języku JavaScript. Eksperymenty zostały przeprowadzone z wykorzystaniem dedykowanej aplikacji testowej, umożliwiającej rzetelny pomiar parametrów wydajnościowych w kontrolowanym środowisku.
	
	Uzyskane wyniki pozwalają określić zależności pomiędzy wyborem protokołów komunikacyjnych, formatów danych oraz technologii wykonawczych a osiąganą wydajnością systemu. Na ich podstawie sformułowano praktyczne wnioski i rekomendacje, które mogą wspierać proces projektowania i optymalizacji nowoczesnych aplikacji webowych oraz architektur rozproszonych.
	
	\begin{center}
		\textbf{Abstract}
	\end{center}
	
	The aim of this thesis is to compare the performance of communication protocols used in web applications and to analyze the impact of data exchange formats on communication efficiency in modern information systems. The study covers protocols such as HTTP/1.1, HTTP/2, GraphQL, gRPC, SOAP, and WebSocket, as well as data formats including JSON, XML, Protocol Buffers, and MessagePack. The analysis focuses on data size, processing time, and cross-system compatibility.
	
	Additionally, the thesis evaluates the impact of using WebAssembly on client-side data processing performance compared to the traditional JavaScript-based approach. The experiments were conducted using a dedicated test application that enabled reliable performance measurements in a controlled environment.
	
	The obtained results allow identifying relationships between the choice of communication protocols, data formats, and execution technologies and the resulting system performance. Based on these findings, practical conclusions and recommendations are formulated to support the design and optimization of modern web applications and distributed architectures.
	
	\noindent\textbf{Słowa kluczowe:}{Rust, WebAssembly, JavaScript, Sieci komputerowe, Protokoły komunikacyjne, Optymalizacja, Serializacja, Deserializacja, Klient--serwer}
\end{abstract}


\onehalfspacing
\thispagestyle{empty}
\tableofcontents


\chapter*{Wstęp}

Dynamiczny rozwój aplikacji webowych oraz rosnące wymagania dotyczące ich skalowalności, responsywności i niezawodności sprawiają, że wydajność komunikacji sieciowej staje się jednym z kluczowych aspektów projektowania nowoczesnych systemów informatycznych. Współczesne aplikacje coraz częściej opierają się na architekturach rozproszonych, w których wymiana danych pomiędzy klientem a serwerem, a także poszczególnymi usługami, odbywa się z wykorzystaniem różnych protokołów komunikacyjnych oraz formatów danych. Wybór odpowiednich technologii w tym zakresie ma bezpośredni wpływ na czas odpowiedzi systemu, obciążenie sieci oraz koszty przetwarzania po obu stronach komunikacji.

Celem niniejszej pracy magisterskiej jest analiza i porównanie wydajności wybranych protokołów komunikacyjnych stosowanych w aplikacjach webowych, a także ocena wpływu formatów wymiany danych oraz technologii wykonywania kodu po stronie klienta na ogólną efektywność systemu. W pracy szczególną uwagę poświęcono porównaniu protokołów HTTP/1.1, HTTP/2, GraphQL, gRPC, SOAP oraz WebSocket, które reprezentują różne podejścia do komunikacji w środowiskach sieciowych. Analizie poddano również popularne formaty danych, takie jak JSON, XML, Protocol Buffers oraz MessagePack, uwzględniając ich rozmiar, czas serializacji i deserializacji oraz kompatybilność pomiędzy różnymi systemami.

Dodatkowym celem pracy jest zbadanie wpływu zastosowania technologii WebAssembly na wydajność przetwarzania danych po stronie klienta w porównaniu z tradycyjnym podejściem opartym na języku JavaScript. W tym kontekście przeprowadzono eksperymenty mające na celu ocenę różnic w czasie wykonywania operacji, zużyciu zasobów oraz potencjalnych korzyściach wynikających z wykorzystania WebAssembly w aplikacjach webowych.

Praca została podzielona na cztery rozdziały.  

W pierwszym rozdziale przedstawiono wprowadzenie do tematyki pracy, obejmujące podstawowe pojęcia związane z komunikacją w aplikacjach webowych oraz uzasadnienie podjęcia badań nad wydajnością protokołów i technologii wykorzystywanych w nowoczesnych systemach informatycznych.  

Rozdział drugi poświęcono porównaniu protokołów komunikacyjnych stosowanych w aplikacjach webowych. Opisano w nim charakterystykę wybranych protokołów, takich jak HTTP/1.1, HTTP/2, GraphQL, gRPC, SOAP, WebSocket Secure oraz WebSocket, a także przedstawiono środowisko badawcze, wykorzystane narzędzia, biblioteki oraz sposób przeprowadzania eksperymentów wydajnościowych.  

W rozdziale trzecim zaprezentowano różnice pomiędzy formatami danych wykorzystywanymi w komunikacji sieciowej. Omówiono takie aspekty jak rozmiar przesyłanych danych, czas przetwarzania oraz kompatybilność pomiędzy systemami na przykładzie formatów JSON, XML, Protocol Buffers oraz MessagePack.  

Czwarty rozdział poświęcono analizie wydajności technologii WebAssembly w porównaniu z JavaScriptem. Przedstawiono w nim wyniki przeprowadzonych eksperymentów oraz ocenę wpływu obu podejść na wydajność aplikacji webowych.  

Podsumowując, praca ma na celu dostarczenie praktycznych i eksperymentalnie potwierdzonych wniosków, które mogą stanowić wsparcie przy wyborze protokołów komunikacyjnych, formatów danych oraz technologii wykonawczych w projektowaniu wydajnych i nowoczesnych aplikacji webowych.

\chapter{Wprowadzenie do problematyki doboru technologii}

\section{Znaczenie doboru technologii w aplikacjach webowych}

Współczesne aplikacje webowe stanowią podstawę funkcjonowania wielu systemów informatycznych wykorzystywanych w biznesie, administracji publicznej oraz życiu codziennym. Rosnąca liczba użytkowników, potrzeba obsługi dużych wolumenów danych oraz wymagania dotyczące niskich czasów odpowiedzi powodują, że zagadnienia związane z wydajnością i skalowalnością systemów stają się kluczowe już na etapie projektowania architektury. Jednym z najważniejszych czynników wpływających na te cechy jest dobór odpowiednich technologii komunikacyjnych oraz formatów wymiany danych.

W architekturach rozproszonych komunikacja pomiędzy komponentami systemu odbywa się za pośrednictwem sieci komputerowej, która wprowadza dodatkowe opóźnienia oraz ograniczenia przepustowości. Niewłaściwy wybór protokołu komunikacyjnego lub formatu danych może prowadzić do nadmiernego obciążenia sieci, zwiększonego zużycia zasobów obliczeniowych oraz pogorszenia doświadczeń użytkownika końcowego. Z tego względu świadomy dobór technologii powinien być oparty nie tylko na ich popularności, lecz przede wszystkim na analizie wymagań danego problemu.

\section{Charakterystyka współczesnych systemów rozproszonych}

Nowoczesne systemy informatyczne coraz częściej projektowane są w oparciu o architektury rozproszone\cite{Systemy_rozporszone}, takie jak architektura klient--serwer, mikroserwisy\cite{Microservice} czy systemy oparte na komunikacji zdarzeniowej\cite{event_driven_architecture}. W tego typu rozwiązaniach poszczególne komponenty systemu działają niezależnie i komunikują się ze sobą za pomocą jasno zdefiniowanych interfejsów.

Taki model projektowy umożliwia łatwiejszą skalowalność oraz rozwój systemu, jednak jednocześnie zwiększa znaczenie wydajnej i niezawodnej komunikacji. Każde wywołanie zdalne wiąże się z kosztami transmisji danych, serializacji\cite{serialization_wiki} i deserializacji\cite{deserialization_wiki} komunikatów oraz przetwarzania po obu stronach połączenia. W praktyce oznacza to, że nawet niewielkie różnice w zastosowanych technologiach mogą prowadzić do zauważalnych różnic w wydajności całego systemu.

\section{Protokoły komunikacyjne jako fundament wymiany danych}

Protokoły komunikacyjne definiują sposób, w jaki dane są przesyłane pomiędzy uczestnikami komunikacji. W aplikacjach webowych powszechnie wykorzystywane są protokoły oparte na rodzinie HTTP, jednak wraz z rozwojem technologii pojawiły się również alternatywne rozwiązania, takie jak gRPC, WebSocket, GraphQL. Każdy z tych protokołów oferuje inne właściwości w zakresie wydajności, sposobu zestawiania połączeń, obsługi strumieniowania danych oraz kompatybilności z istniejącą infrastrukturą.

Dobór protokołu komunikacyjnego powinien uwzględniać charakter wymiany danych, częstotliwość komunikacji, wymagania dotyczące opóźnień oraz środowisko uruchomieniowe aplikacji. Przykładowo, protokoły oparte na długotrwałych połączeniach mogą być bardziej efektywne w aplikacjach czasu rzeczywistego, natomiast klasyczne podejście żądanie--odpowiedź sprawdza się w prostszych scenariuszach komunikacyjnych.

\section{Uzasadnienie podjęcia badań}

Różnorodność dostępnych protokołów komunikacyjnych, formatów danych oraz technologii wykonawczych sprawia, że projektanci systemów informatycznych stają przed trudnym zadaniem wyboru najbardziej odpowiednich rozwiązań. Brak jednoznacznych odpowiedzi oraz silne uzależnienie wyników od kontekstu zastosowania powodują, że decyzje te często podejmowane są intuicyjnie.

Celem niniejszej pracy jest dostarczenie empirycznych danych oraz praktycznych wniosków, które mogą wspierać proces podejmowania decyzji technologicznych. Przeprowadzone analizy i eksperymenty pozwalają lepiej zrozumieć zależności pomiędzy wyborem technologii a wydajnością aplikacji webowych.

\section{Słownik pojęć i skrótów}

\begin{description}
	\item[\textbf{API} (\textit{Application Programming Interface})] --
	Zbiór reguł, definicji oraz mechanizmów umożliwiających komunikację pomiędzy różnymi komponentami oprogramowania\cite{api_wiki}.
	
	\item[\textbf{HTTP} (\textit{Hypertext Transfer Protocol})] --
	Protokół komunikacyjny wykorzystywany do przesyłania danych w sieci WWW w modelu klient--serwer.
	
	\item[\textbf{WebSocket}] --
	Protokół umożliwiający utrzymywanie stałego, dwukierunkowego połączenia pomiędzy klientem a serwerem, wykorzystywany do komunikacji w czasie rzeczywistym.
	
	\item[\textbf{gRPC}] --
	Wysokowydajny framework komunikacyjny typu RPC, oparty na protokole HTTP/2 oraz binarnym formacie \textit{Protocol Buffers}.
	
	\item[\textbf{RPC} (\textit{Remote Procedure Call})] --
	Mechanizm komunikacji międzyprocesowej umożliwiający wykonywanie procedur lub funkcji w zdalnym systemie komputerowym tak, jakby były wywoływane lokalnie\cite{RPC}.
	
	\item[\textbf{Protocol Buffers} (\textit{Protobuf})] --
	Binarny format serializacji danych opracowany przez firmę Google, charakteryzujący się kompaktową reprezentacją oraz efektywnym przetwarzaniem, wykorzystywany m.in. w komunikacji gRPC.
	
	\item[\textbf{SOAP} (\textit{Simple Object Access Protocol})] --
	Protokół komunikacyjny oparty na wymianie komunikatów XML, stosowany w architekturach usług sieciowych.
	
	\item[\textbf{MQTT} (\textit{Message Queuing Telemetry Transport})] --
	Lekki protokół komunikacyjny typu publish--subscribe, przeznaczony do systemów o ograniczonych zasobach oraz komunikacji w sieciach o niskiej przepustowości\cite{MQTT}.
	
	\item[\textbf{JSON} (\textit{JavaScript Object Notation})] --
	Tekstowy format wymiany danych oparty na składni JavaScript, charakteryzujący się czytelnością dla człowieka oraz łatwością parsowania przez maszyny.
	
	\item[\textbf{MessagePack}] --
	Binarny format serializacji danych zaprojektowany jako wydajniejsza alternatywa dla JSON, oferujący mniejszy rozmiar oraz szybsze przetwarzanie przy zachowaniu podobnej elastyczności.
	
	\item[\textbf{Apache Avro}] --
	System serializacji danych opracowany w ramach projektu Apache Hadoop, wykorzystujący schematy JSON do definiowania struktury danych oraz zapewniający kompaktową reprezentację binarną\cite{avro}.
	
	\item[\textbf{BSON} (\textit{Binary JSON})] --
	Binarny format serializacji dokumentów wzorowany na JSON, wykorzystywany m.in. w bazie danych MongoDB, rozszerzający JSON o dodatkowe typy danych oraz umożliwiający efektywne przechowywanie i przetwarzanie\cite{BSON}.
	
	\item[\textbf{Serializacja}] --
	Proces przekształcania struktury danych do postaci umożliwiającej jej zapis lub transmisję pomiędzy systemami\cite{serialization_wiki}.
	
	\item[\textbf{Deserializacja}] --
	Proces odtwarzania pierwotnej struktury danych na podstawie jej zserializowanej reprezentacji\cite{deserialization_wiki}.
	
	\item[\textbf{XML} (\textit{eXtensible Markup Language})] --
	Rozszerzalny język znaczników wykorzystywany do opisu i strukturyzacji danych tekstowych.
	
	\item[\textbf{Mikroserwis}] --
	Architektoniczny wzorzec projektowy, w którym aplikacja składa się z wielu niezależnych, luźno powiązanych usług, z których każda realizuje konkretną funkcjonalność biznesową i może być rozwijana oraz wdrażana niezależnie.
	
	\item[\textbf{Systemy rozproszone}] --
	Zbiór niezależnych komponentów obliczeniowych połączonych siecią, które współpracują ze sobą w celu realizacji wspólnego zadania, prezentując się użytkownikowi jako jeden spójny system.
	
	\item[\textbf{Komunikacja zdarzeniowa} (\textit{Event-Driven Architecture})] --
	Architektura oprogramowania, w której komponenty systemu komunikują się poprzez generowanie, wykrywanie i reagowanie na zdarzenia, umożliwiając luźne powiązanie oraz asynchroniczną wymianę informacji.
	
	\item[\textbf{Strumieniowanie danych}] --
	Technika przetwarzania i przesyłania danych w sposób ciągły, bez konieczności oczekiwania na kompletny zbiór danych\cite{stream}.
	
	\item[\textbf{Head-of-Line Blocking} (\textit{HOL})] --
	Zjawisko polegające na blokowaniu przetwarzania kolejnych żądań przez opóźnienie jednego z nich w ramach tego samego połączenia\cite{HOL}.
	
	\item[\textbf{WebAssembly} (\textit{Wasm})] --
	Binarny format wykonywalnego kodu umożliwiający uruchamianie aplikacji o wysokiej wydajności w środowisku przeglądarek internetowych.
	
	\item[\textbf{TLS} (\textit{Transport Layer Security})] --
	Protokół kryptograficzny zapewniający poufność, integralność oraz uwierzytelnianie danych przesyłanych w sieci komputerowej\cite{tls}.
	
	\item[\textbf{Architektura klient--serwer}] --
	Model architektoniczny, w którym klient inicjuje żądania, a serwer przetwarza je i zwraca odpowiedzi\cite{client_server_meaning}.
	
	\item[\textbf{Silnik JavaScript V8}] --
	Wysokowydajny silnik JavaScript opracowany przez firmę Google, wykorzystywany m.in. w przeglądarce Google Chrome oraz środowisku Node.js\cite{Jv8}.
	
	\item[\textbf{DOM} (\textit{Document Object Model})] --
	Model obiektowy dokumentu HTML lub XML, który reprezentuje jego strukturę w postaci drzewa obiektów, umożliwiając programom (np. skryptom JavaScript) dynamiczny dostęp do zawartości, struktury oraz stylów strony internetowej\cite{DOM}.
	
	\item[\textbf{Web Worker}] --
	Mechanizm platformy webowej umożliwiający uruchamianie skryptów JavaScript w tle, w osobnym wątku względem głównego wątku interfejsu użytkownika, co pozwala na wykonywanie kosztownych obliczeń bez blokowania renderowania strony oraz interakcji użytkownika. Web Worker nie ma bezpośredniego dostępu do drzewa DOM, a komunikacja z głównym wątkiem odbywa się za pomocą asynchronicznego przesyłania komunikatów\cite{web_worker}.
	
\end{description}

\chapter{Technologie wykorzystane w projekcie}

\section{Rust}

Rust jest nowoczesnym językiem programowania systemowego, który kładzie duży nacisk na bezpieczeństwo pamięci, wydajność oraz wielowątkowość. Dzięki mechanizmowi własności (ownership) oraz statycznej analizie błędów w czasie kompilacji, Rust minimalizuje ryzyko występowania błędów takich jak wycieki pamięci czy dereferencja pustych wskaźników. Rust zdobył popularność również dzięki nowoczesnemu systemowi typów i możliwości pisania kodu niskopoziomowego bez rezygnacji z bezpieczeństwa \cite{rust}.

Wybór języka Rust był podyktowany potrzebą uzyskania niskich czasów odpowiedzi oraz stabilności działania aplikacji przy dużej liczbie równoległych zapytań. Dodatkowo paczki wykorzystane do badań nie posiadały zbędnych funkcjonalności, co umożliwiło skupienie się na analizie wyników i minimalnej implementacji rozwiązań.

Według Tiobe Index, Rust zyskał największą dotychczas popularność 1 stycznia 2026 roku.

\begin{figure}[h]
	\centering
	\includegraphics[width=0.7\textwidth]{images/rust_tiobe_index}
	\caption{Pozycja języka Rust w rankingu Tiobe Index.}
	\label{fig:rust_tiobe}
\end{figure}

Jak widać na Rysunku~\ref{fig:rust_tiobe}, Rust zyskuje coraz większą popularność wśród programistów, co potwierdza rosnące zainteresowanie nim w przemyśle i projektach open source.

\section{JavaScript}

JavaScript jest językiem skryptowym powszechnie wykorzystywanym do tworzenia aplikacji webowych \cite{javascript}. W projekcie został użyty głównie jako punkt odniesienia dla tradycyjnych interfejsów użytkownika w porównaniu z technologią WebAssembly. Dzięki swojej uniwersalności i dużemu ekosystemowi bibliotek, JavaScript nadal pozostaje podstawowym językiem front-endowym.

\section{TypeScript}

TypeScript jest statycznie typowanym nadzbiorem języka JavaScript, rozwijanym przez firmę Microsoft. Rozszerza on JavaScript o system typów oraz mechanizmy weryfikacji poprawności kodu na etapie kompilacji, co znacząco ułatwia tworzenie, utrzymanie i rozwój większych aplikacji \cite{typesciript}.

\section{Framework Actix-web}

Actix-web jest asynchronicznym frameworkiem webowym dla języka Rust, opartym na modelu aktorów. Charakteryzuje się wysoką wydajnością oraz niskim narzutem czasowym, co sprawia, że jest jedną z najszybszych opcji w ekosystemie Rust \cite{actix}.

Framework Actix-web został wybrany ze względu na:
\begin{itemize}[itemsep=0.2em, parsep=0pt, topsep=0pt]
	\item wysoką wydajność potwierdzoną testami benchmarkowymi,
	\item dobrą integrację z ekosystemem Rust,
	\item wsparcie dla programowania asynchronicznego.
\end{itemize}

\begin{figure}[h]
	\centering
	\includegraphics[width=0.7\textwidth]{images/rust_actix_web}
	\caption{Framework Actix-web w porównaniu z innymi frameworkami.}
	\label{fig:rust_actix_web}
\end{figure}

Jak widać na Rysunku~\ref{fig:rust_actix_web}, Actix wypada bardzo korzystnie w kontekście wydajności i obsługi dużego ruchu, co jest istotne dla aplikacji wymagających niskich czasów odpowiedzi.

\section{Framework Tonic}

Tonic jest frameworkiem do implementacji gRPC w języku Rust, opartym na bibliotece \texttt{tokio} oraz protokole HTTP/2 \cite{tonic}. W projekcie został wykorzystany do:
\begin{itemize}[itemsep=0.2em, parsep=0pt, topsep=0pt]
	\item implementacji serwera gRPC,
	\item definiowania kontraktów komunikacyjnych w postaci plików \texttt{.proto},
	\item realizacji wydajnej komunikacji klient–serwer.
\end{itemize}

Framework Tonic umożliwił stworzenie stabilnej i szybkiej komunikacji typu RPC, co było kluczowe dla testów wydajnościowych i porównawczych z GraphQL.

\section{tungstenite-rs}

\texttt{tungstenite-rs} jest biblioteką w Rust umożliwiającą obsługę protokołu WebSocket. Została wykorzystana do \cite{tungstenite}:
\begin{itemize}[itemsep=0.2em, parsep=0pt, topsep=0pt]
	\item obsługi komunikacji w czasie rzeczywistym,
	\item przesyłania danych bez konieczności inicjowania nowych połączeń HTTP,
	\item testów alternatywnych modeli komunikacji.
\end{itemize}

\section{h2load}

\texttt{h2load} jest narzędziem służącym do testowania wydajności aplikacji korzystających z protokołu HTTP/2 \cite{h2load}. W projekcie zostało wykorzystane do:
\begin{itemize}[itemsep=0.2em, parsep=0pt, topsep=0pt]
	\item generowania dużej liczby równoległych zapytań,
	\item pomiaru czasów odpowiedzi serwera,
	\item analizy zachowania systemu pod obciążeniem.
\end{itemize}

\section{Python}

Python jest językiem wysokiego poziomu, który w projekcie został użyty głównie pomocniczo, do:
\begin{itemize}[itemsep=0.2em, parsep=0pt, topsep=0pt]
	\item analizy wyników testów wydajnościowych,
	\item generowania wykresów porównawczych,
	\item przetwarzania danych pomiarowych.
\end{itemize}

Python pozostaje jednym z najpopularniejszych języków programowania \cite{python}, co obrazuje Rysunek~\ref{fig:python_tiobe}.

\begin{figure}[h]
	\centering
	\includegraphics[width=0.7\textwidth]{images/python_tiobe_index}
	\caption{Pozycja języka Python w rankingu Tiobe Index.}
	\label{fig:python_tiobe}
\end{figure}

\section{GraphQL}

GraphQL jest językiem zapytań do API, pozwalającym klientowi precyzyjnie określić, jakie dane są potrzebne \cite{gql}. W projekcie został wykorzystany do:
\begin{itemize}[itemsep=0.2em, parsep=0pt, topsep=0pt]
	\item realizacji zapytań o złożone struktury danych,
	\item testów wydajnościowych i porównawczych.
\end{itemize}

\section{gRPC}

gRPC jest mechanizmem komunikacji RPC opartym na protokole HTTP/2 oraz formacie binarnym Protocol Buffers. W projekcie zostało wykorzystane do:
\begin{itemize}[itemsep=0.2em, parsep=0pt, topsep=0pt]
	\item komunikacji między komponentami systemu,
	\item przeprowadzania testów wydajnościowych,
\end{itemize}

\section{WebAssembly}

WebAssembly (Wasm) jest binarnym formatem wykonywalnym, pozwalającym uruchamiać kod napisany w Rust bezpośrednio w przeglądarce. W projekcie zostało wykorzystane do:
\begin{itemize}[itemsep=0.2em, parsep=0pt, topsep=0pt]
	\item uruchamiania części logiki aplikacji po stronie klienta,
	\item eksperymentalnego porównania wydajności z klasycznym JavaScriptem,
	\item zwiększenia wydajności aplikacji webowej.
\end{itemize}
\section{rumqttc}

\texttt{rumqttc} jest asynchroniczną biblioteką w języku Rust służącą do obsługi protokołu MQTT (Message Queuing Telemetry Transport). Umożliwia ona implementację klientów MQTT, oferując niskie opóźnienia oraz wysoką wydajność komunikacji \cite{rumqttc}.

W projekcie biblioteka \texttt{rumqttc} została wykorzystana do:
\begin{itemize}[itemsep=0.2em, parsep=0pt, topsep=0pt]
	\item testów komunikacji typu publish–subscribe,
	\item analizy wydajności lekkich protokołów komunikacyjnych,
	\item porównania MQTT z innymi mechanizmami wymiany danych.
\end{itemize}

\section{wasm-pack}

\texttt{wasm-pack} jest narzędziem wspierającym proces kompilacji projektów napisanych w języku Rust do formatu WebAssembly. Automatyzuje on generowanie pakietów WASM, integrację z ekosystemem JavaScript oraz publikację modułów.

W projekcie \texttt{wasm-pack} został wykorzystany do:
\begin{itemize}[itemsep=0.2em, parsep=0pt, topsep=0pt]
	\item kompilacji modułów Rust do WebAssembly,
	\item przygotowania interfejsów JavaScript do komunikacji z WASM,
	\item budowy aplikacji demonstracyjnych uruchamianych w przeglądarce.
\end{itemize}

\section{Juniper}

\texttt{Juniper} jest biblioteką w języku Rust służącą do implementacji serwerów GraphQL. Umożliwia definiowanie schematów, zapytań oraz mutacji w sposób silnie typowany, zgodny z paradygmatami języka Rust \cite{juniper}.

W projekcie biblioteka \texttt{Juniper} została użyta do:
\begin{itemize}[itemsep=0.2em, parsep=0pt, topsep=0pt]
	\item implementacji API GraphQL,
	\item realizacji zapytań o złożone struktury danych,
	\item testów wydajnościowych komunikacji GraphQL.
\end{itemize}

\section{Apache Avro}

Apache Avro jest binarnym formatem serializacji danych opartym na schematach, wykorzystywanym głównie w systemach przetwarzania danych i architekturach rozproszonych. Zapewnia kompaktową reprezentację danych oraz możliwość ewolucji schematów \cite{avro}.

W projekcie Apache Avro został wykorzystany do:
\begin{itemize}[itemsep=0.2em, parsep=0pt, topsep=0pt]
	\item testów wydajności serializacji i deserializacji danych,
	\item porównania z innymi formatami binarnymi i tekstowymi,
	\item analizy rozmiaru wynikowych struktur danych.
\end{itemize}

\section{BSON}

BSON (Binary JSON) jest binarną reprezentacją formatu JSON, zaprojektowaną z myślą o efektywnej serializacji danych. Jest powszechnie stosowany m.in. w bazach danych MongoDB \cite{bson}.

W projekcie format BSON został wykorzystany do:
\begin{itemize}[itemsep=0.2em, parsep=0pt, topsep=0pt]
	\item testów wydajności serializacji danych,
	\item porównania binarnych i tekstowych formatów wymiany danych,
	\item analizy narzutu pamięciowego.
\end{itemize}

\section{Serde}

\texttt{Serde} jest biblioteką w języku Rust służącą do serializacji i deserializacji struktur danych \cite{serde}. Zapewnia zunifikowany interfejs obsługujący wiele formatów, takich jak JSON, BSON, XML czy MessagePack.

W projekcie \texttt{Serde} została wykorzystana jako:
\begin{itemize}[itemsep=0.2em, parsep=0pt, topsep=0pt]
	\item warstwa abstrakcji dla procesów serializacji danych,
	\item wspólna baza dla testów różnych formatów danych,
	\item narzędzie upraszczające implementację testów porównawczych.
\end{itemize}

\section{Prost}

\texttt{Prost} jest biblioteką w języku Rust służącą do obsługi formatu Protocol Buffers. Umożliwia generowanie kodu Rust na podstawie plików \texttt{.proto} oraz efektywną serializację danych binarnych \cite{prost}.

W projekcie \texttt{Prost} została wykorzystana do:
\begin{itemize}[itemsep=0.2em, parsep=0pt, topsep=0pt]
	\item obsługi komunikacji gRPC,
	\item serializacji danych w formacie Protocol Buffers,
	\item testów wydajności komunikacji RPC.
\end{itemize}

\section{quick-xml}

\texttt{quick-xml} jest szybką biblioteką w języku Rust przeznaczoną do parsowania i generowania dokumentów XML. Charakteryzuje się niskim narzutem pamięciowym oraz wysoką wydajnością \cite{qXML}.

W projekcie biblioteka \texttt{quick-xml} została użyta do:
\begin{itemize}[itemsep=0.2em, parsep=0pt, topsep=0pt]
	\item testów serializacji i deserializacji danych w formacie XML,
	\item porównania XML z innymi formatami wymiany danych,
	\item analizy wydajności przetwarzania danych tekstowych.
\end{itemize}
\chapter{Użyte narzędzia programistyczne}

\section{Git}
Git jest rozproszonym systemem kontroli wersji, którego głównym celem jest zarządzanie historią zmian w kodzie źródłowym projektu programistycznego \cite{git}. Narzędzie to umożliwia rejestrowanie kolejnych wersji plików, analizę wprowadzonych modyfikacji oraz powrót do wcześniejszych stanów projektu. Dzięki rozproszonej architekturze każdy użytkownik posiada pełną kopię repozytorium wraz z całą historią zmian, co zwiększa niezawodność oraz elastyczność pracy zespołowej.

Git oferuje mechanizmy takie jak gałęzie (ang. \textit{branches}) oraz scalanie zmian (ang. \textit{merge}), które pozwalają na równoległą pracę nad różnymi funkcjonalnościami bez ingerencji w główną wersję projektu. System ten jest szeroko stosowany zarówno w małych projektach indywidualnych, jak i w dużych przedsięwzięciach komercyjnych oraz otwartoźródłowych.

\section{Zed}
Zed jest nowoczesnym edytorem kodu źródłowego zaprojektowanym z myślą o wysokiej wydajności, niskich opóźnieniach oraz współczesnych potrzebach programistów \cite{zed}. Narzędzie to zostało stworzone przy wykorzystaniu języka Rust, co przekłada się na bezpieczeństwo pamięci oraz wysoką responsywność interfejsu użytkownika.

Edytor oferuje wsparcie dla wielu języków programowania, inteligentne podpowiedzi kodu, integrację z systemami kontroli wersji oraz możliwość pracy zespołowej w czasie rzeczywistym. Zed kładzie duży nacisk na minimalizm interfejsu oraz płynność pracy, co czyni go atrakcyjną alternatywą dla bardziej rozbudowanych środowisk programistycznych.

\section{Visual Studio Code}
Visual Studio Code jest wieloplatformowym edytorem kodu źródłowego rozwijanym przez firmę Microsoft \cite{vscode}. Narzędzie to łączy w sobie prostotę edytora tekstu z funkcjonalnością zintegrowanego środowiska programistycznego (IDE). Dzięki rozbudowanemu systemowi rozszerzeń możliwe jest dostosowanie edytora do pracy z niemal dowolnym językiem programowania oraz frameworkiem.

Visual Studio Code oferuje funkcje takie jak podświetlanie składni, inteligentne uzupełnianie kodu, debugowanie aplikacji oraz integrację z systemem Git. Jego popularność wynika z dużej elastyczności, aktywnej społeczności oraz regularnych aktualizacji, co czyni go jednym z najczęściej wybieranych narzędzi programistycznych.

\section{NPM}
NPM (Node Package Manager) jest menedżerem pakietów przeznaczonym dla środowiska Node.js \cite{npm}. Jego głównym zadaniem jest zarządzanie zależnościami projektu, w tym instalowanie, aktualizowanie oraz usuwanie bibliotek zewnętrznych. NPM umożliwia również publikowanie własnych pakietów w publicznym repozytorium, co wspiera ponowne wykorzystanie kodu.

Centralnym elementem działania NPM jest plik \texttt{package.json}, w którym definiowane są informacje o projekcie, jego zależnościach oraz skryptach automatyzujących typowe zadania, takie jak budowanie aplikacji czy uruchamianie testów. Narzędzie to stanowi podstawowy element ekosystemu JavaScript i jest powszechnie wykorzystywane w projektach frontendowych oraz backendowych.

\section{Cargo}
Cargo jest oficjalnym narzędziem do zarządzania projektami w języku Rust \cite{cargo}. Odpowiada ono za proces kompilacji kodu źródłowego, pobieranie i zarządzanie zależnościami oraz uruchamianie testów jednostkowych. Cargo integruje się bezpośrednio z kompilatorem Rust, zapewniając spójność i automatyzację procesu budowania aplikacji.

Konfiguracja projektu realizowana jest za pomocą pliku \texttt{Cargo.toml}, który zawiera informacje o projekcie, wersjach bibliotek oraz ustawieniach kompilacji. Dzięki Cargo proces tworzenia aplikacji w języku Rust jest uproszczony i ustandaryzowany, co sprzyja utrzymaniu wysokiej jakości kodu oraz jego skalowalności.

\section{Testy wydajnościowe}

W ramach przygotowanego środowiska testowego przeprowadzono szereg eksperymentów mających na celu obiektywną ocenę wydajności różnych mechanizmów komunikacji oraz formatów serializacji danych. Poniżej przedstawiono szczegółowy opis metodologii testowania poszczególnych grup aplikacji.

\subsection{Aplikacje zawarte w katalogu communication-mechanisms}

Dla każdego zaimplementowanego protokołu komunikacji przeprowadzono cztery kategorie testów wydajnościowych, pozwalających na kompleksową ocenę charakterystyk przesyłania danych:

\begin{enumerate}
	\item \textbf{Transmisja małych pakietów danych} -- test polegający na cyklicznym odpytywaniu serwera o dane o rozmiarze 1~KB. Pomiar ten pozwala ocenić narzut protokołu oraz czas odpowiedzi dla typowych żądań zawierających niewielką ilość informacji.
	
	\item \textbf{Transmisja dużych pakietów danych} -- test analogiczny do poprzedniego, z tą różnicą, że rozmiar przesyłanych danych wynosi 1~MB. Eksperyment ten umożliwia ocenę wydajności protokołów w scenariuszu transferu większych zasobów.
	
	\item \textbf{Head-of-Line Blocking} -- test weryfikujący występowanie zjawiska blokowania głowy kolejki, w którym opóźnienie w przetwarzaniu jednego żądania wpływa na obsługę kolejnych żądań w ramach tego samego połączenia. Pomiar ten jest szczególnie istotny dla protokołów wykorzystujących multipleksowanie strumieni.
	
	\item \textbf{Strumieniowanie danych} -- test przeprowadzany wyłącznie dla protokołów wspierających tryb strumieniowy (streaming). Ocenie podlega efektywność przesyłania danych w sposób ciągły, bez konieczności oczekiwania na kompletną odpowiedź serwera.
\end{enumerate}

\subsection{Aplikacje zawarte w katalogu data-serialization-formats}

Dla każdej z zaimplementowanych bibliotek serializacji danych przeprowadzono pomiary wydajności na podstawie ustandaryzowanej struktury danych. W celu zapewnienia porównywalności wyników, wszystkie testy wykorzystują identyczny zestaw danych testowych składający się z kolekcji 1000 obiektów użytkowników wraz z metadanymi.

Struktura danych testowych jest generowana w następujący sposób:

\begin{lstlisting}[style=rustcode, caption={Struktura danych wykorzystana w testach serializacji.}, label={lst:user_struct}]
	let count = 1000;
	let users: Vec<User> = (0..count)
	.map(|i| User {
		id: i as i64,
		name: format!("User {}", i),
		email: format!("user{}@example.com", i),
		age: 20 + (i % 50) as i32,
		is_active: i % 2 == 0,
		tags: vec![
		"tag1".to_string(),
		"tag2".to_string(),
		"tag3".to_string(),
		],
	})
	.collect();
	
	UserCollection {
		users,
		metadata: Metadata {
			version: "1.0.0".to_string(),
			created_at: "2025-01-23T00:00:00Z".to_string(),
			total_count: count,
		},
	}
\end{lstlisting}

Każdy obiekt użytkownika zawiera kompletny zestaw atrybutów: identyfikator liczbowy, nazwę, adres email, wiek, status aktywności oraz listę trzech tagów. Dodatkowo, kolekcja jest wzbogacona o metadane zawierające wersję struktury danych, znacznik czasu utworzenia oraz łączną liczbę elementów. Taka konstrukcja zapewnia reprezentatywny zestaw różnorodnych typów danych (liczby całkowite, ciągi znaków, wartości logiczne, kolekcje) występujących w rzeczywistych zastosowaniach.

Dla każdego formatu serializacji mierzono następujące parametry wydajnościowe:

\begin{enumerate}
	\item \textbf{Czas serializacji} -- czas potrzebny na przekształcenie struktury danych z reprezentacji obiektowej języka Rust do formatu serializowanego. Pomiar wyrażony w mikrosekundach ($\mu$s) i uśredniony na podstawie wielokrotnych iteracji.
	
	\item \textbf{Czas deserializacji} -- czas wymagany do odtworzenia struktury obiektowej z danych w formacie serializowanym. Pomiar wyrażony w mikrosekundach ($\mu$s) i uśredniony na podstawie wielokrotnych iteracji.
	
	\item \textbf{Rozmiar zserializowanych danych} -- całkowita wielkość danych po procesie serializacji, wyrażona w bajtach (B). Parametr ten pozwala ocenić efektywność kompresji oraz narzut protokołu dla różnych formatów wymiany danych.
\end{enumerate}

Wykorzystanie identycznego zestawu danych testowych dla wszystkich formatów serializacji umożliwia obiektywne porównanie ich wydajności oraz charakterystyk w zakresie rozmiaru wynikowych struktur danych.

\subsection{Testy przeglądarkowe}

Testy przeglądarkowe służą do porównania wydajności implementacji algorytmów w czystym języku JavaScript oraz w technologii WebAssembly. Eksperyment polega na równoległym uruchomieniu identycznych funkcjonalności w obu środowiskach wykonawczych i pomiarze czasu ich wykonania.

Testowane są następujące scenariusze:

\begin{itemize}
	\item \textbf{Implementacja JavaScript} -- natywny kod uruchamiany bezpośrednio przez silnik JavaScript przeglądarki,
	\item \textbf{Implementacja WebAssembly} -- kod napisany w języku Rust, skompilowany do formatu WASM za pomocą narzędzia \texttt{wasm-pack} i wykonywany w środowisku przeglądarki.
\end{itemize}

Porównanie obu implementacji pozwala na obiektywną ocenę potencjalnych korzyści wydajnościowych wynikających z zastosowania technologii WebAssembly w aplikacjach webowych wymagających intensywnych obliczeń.

\chapter*{Zakończenie}



\bibliographystyle{plain}
\bibliography{literatura}


% spis rysunków
\listoffigures
% spis listingów 
\lstlistoflistings


% załączniki (opcjonalnie):
\appendix
%\chapter{Tytuł załącznika jeden}
%Treść załącznika jeden.

%\chapter{Tytuł załącznika dwa}
%Treść załącznika dwa.

%\bibliographystyle{plain}
%\bibliography{literatura}

%\printbibliography[type=book,title={Bibliografia - książki}]
%\printbibliography[type=misc,title={Bibliografia - strony internetowe}]

% spis tabel (jeżeli jest potrzebny):
%\listoftables

% spis rysunków (jeżeli jest potrzebny):
%\listoffigures

% spis listingów (jeżeli jest potrzebny):
%\lstlistoflistings

\end{document}
