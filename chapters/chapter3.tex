\chapter{Użyte narzędzia programistyczne}

\section{Git}
\textbf{Git} jest rozproszonym systemem kontroli wersji, którego głównym celem jest zarządzanie historią zmian w kodzie źródłowym projektu programistycznego\cite{git}. Narzędzie to umożliwia rejestrowanie kolejnych wersji plików, analizę wprowadzonych modyfikacji oraz powrót do wcześniejszych stanów projektu. Dzięki rozproszonej architekturze każdy użytkownik posiada pełną kopię repozytorium wraz z całą historią zmian, co zwiększa niezawodność oraz elastyczność pracy zespołowej.

Git oferuje mechanizmy takie jak gałęzie (ang. \textit{branches}) oraz scalanie zmian (ang. \textit{merge}), które pozwalają na równoległą pracę nad różnymi funkcjonalnościami bez ingerencji w główną wersję projektu. System ten jest szeroko stosowany zarówno w małych projektach indywidualnych, jak i w dużych przedsięwzięciach komercyjnych oraz otwartoźródłowych.

\section{Zed}
\textbf{Zed} jest nowoczesnym edytorem kodu źródłowego zaprojektowanym z myślą o wysokiej wydajności, niskich opóźnieniach oraz współczesnych potrzebach programistów\cite{zed}. Narzędzie to zostało stworzone przy wykorzystaniu języka \texttt{Rust}, co przekłada się na bezpieczeństwo pamięci oraz wysoką responsywność interfejsu użytkownika.

Edytor oferuje wsparcie dla wielu języków programowania, inteligentne podpowiedzi kodu, integrację z systemami kontroli wersji oraz możliwość pracy zespołowej w czasie rzeczywistym. Zed kładzie duży nacisk na minimalizm interfejsu oraz płynność pracy, co czyni go atrakcyjną alternatywą dla bardziej rozbudowanych środowisk programistycznych.

\section{Visual Studio Code}
\textbf{Visual Studio Code} jest wieloplatformowym edytorem kodu źródłowego rozwijanym przez firmę Microsoft\cite{vscode}. Narzędzie to łączy w sobie prostotę edytora tekstu z funkcjonalnością zintegrowanego środowiska programistycznego (\texttt{IDE}). Dzięki rozbudowanemu systemowi rozszerzeń możliwe jest dostosowanie edytora do pracy z niemal dowolnym językiem programowania oraz frameworkiem.

Visual Studio Code oferuje funkcje takie jak podświetlanie składni, inteligentne uzupełnianie kodu, debugowanie aplikacji oraz integrację z systemem \texttt{Git}. Jego popularność wynika z dużej elastyczności, aktywnej społeczności oraz regularnych aktualizacji, co czyni go jednym z najczęściej wybieranych narzędzi programistycznych.

\section{Bun}

\textbf{Bun} jest nowoczesnym środowiskiem uruchomieniowym dla języka \texttt{JavaScript}, \texttt{TypeScript} oraz \texttt{WebAssembly}, pełniącym jednocześnie rolę menedżera pakietów, narzędzia do budowania aplikacji oraz serwera uruchomieniowego\cite{bun}. Jego głównym celem jest zwiększenie wydajności i uproszczenie procesu tworzenia aplikacji webowych poprzez integrację funkcjonalności, które w tradycyjnym ekosystemie \texttt{Node.js}\cite{node.js} realizowane są przez wiele odrębnych narzędzi.

Jednym z kluczowych elementów środowiska Bun jest wbudowany menedżer pakietów, stanowiący alternatywę dla NPM. Umożliwia on instalowanie, aktualizowanie oraz usuwanie zależności projektowych, a także korzystanie z pakietów dostępnych w rejestrze NPM, przy jednoczesnym zachowaniu znacznie krótszych czasów instalacji. Konfiguracja projektu opiera się na pliku \texttt{package.json}, analogicznie do standardowego ekosystemu JavaScript, natomiast dodatkowym elementem jest plik \texttt{bun.lock}, zapewniający deterministyczność i powtarzalność procesu budowania aplikacji.

Bun obsługuje również natywne uruchamianie kodu \texttt{TypeScript} bez konieczności wcześniejszej transpilacji oraz oferuje wbudowany system budowania i bundlowania zasobów. Dzięki temu środowisko to może pełnić rolę kompleksowej platformy deweloperskiej, szczególnie przydatnej w aplikacjach frontendowych, backendowych oraz projektach wykorzystujących technologię \texttt{WebAssembly}.

\section{Cargo}
\textbf{Cargo} jest oficjalnym narzędziem do zarządzania projektami w języku \texttt{Rust}\cite{cargo}. Odpowiada ono za proces kompilacji kodu źródłowego, pobieranie i zarządzanie zależnościami oraz uruchamianie testów jednostkowych. Cargo integruje się bezpośrednio z kompilatorem Rust, zapewniając spójność i automatyzację procesu budowania aplikacji.

Konfiguracja projektu realizowana jest za pomocą pliku \texttt{Cargo.toml}, który zawiera informacje o projekcie, wersjach bibliotek oraz ustawieniach kompilacji. Dzięki Cargo proces tworzenia aplikacji w języku Rust jest uproszczony i ustandaryzowany, co sprzyja utrzymaniu wysokiej jakości kodu oraz jego skalowalności.
