\chapter{Technologie wykorzystane w projekcie}

\section{Rust}

\textbf{Rust} jest nowoczesnym językiem programowania systemowego, który kładzie duży nacisk na bezpieczeństwo pamięci, wydajność oraz wielowątkowość. Dzięki mechanizmowi własności (ownership) oraz statycznej analizie błędów w czasie kompilacji, Rust minimalizuje ryzyko występowania błędów takich jak wycieki pamięci czy dereferencja pustych wskaźników. Rust zdobył popularność również dzięki nowoczesnemu systemowi typów i możliwości pisania kodu niskopoziomowego bez rezygnacji z bezpieczeństwa\cite{rust}.

Wybór języka Rust był podyktowany potrzebą uzyskania niskich czasów odpowiedzi oraz stabilności działania aplikacji przy dużej liczbie równoległych zapytań. Dodatkowo paczki wykorzystane do badań nie posiadały zbędnych funkcjonalności, co umożliwiło skupienie się na analizie wyników i minimalnej implementacji rozwiązań.

Według Tiobe Index, Rust zyskał największą dotychczas popularność 1 stycznia 2026 roku.

\begin{figure}[h]
	\centering
	\includegraphics[width=0.8\textwidth]{images/rust_tiobe_index}
	\caption{Pozycja języka Rust w rankingu Tiobe Index.}
	\label{fig:rust_tiobe}
\end{figure}

Jak widać na Rysunku~\ref{fig:rust_tiobe}, Rust zyskuje coraz większą popularność wśród programistów, co potwierdza rosnące zainteresowanie nim w przemyśle i projektach open source.

\section{JavaScript}

\textbf{JavaScript} jest językiem skryptowym powszechnie wykorzystywanym do tworzenia aplikacji webowych\cite{javascript}. W projekcie został użyty głównie jako punkt odniesienia dla tradycyjnych interfejsów użytkownika w porównaniu z technologią WebAssembly. Dzięki swojej uniwersalności i dużemu ekosystemowi bibliotek, JavaScript nadal pozostaje podstawowym językiem front-endowym.

\section{TypeScript}

\textbf{TypeScript} jest statycznie typowanym nadzbiorem języka JavaScript, rozwijanym przez firmę Microsoft. Rozszerza on JavaScript o system typów oraz mechanizmy weryfikacji poprawności kodu na etapie kompilacji, co znacząco ułatwia tworzenie, utrzymanie i rozwój większych aplikacji\cite{typesciript}.

\section{Framework Actix-web}

\textbf{Actix-web} jest asynchronicznym frameworkiem webowym dla języka Rust, opartym na modelu aktorów. Charakteryzuje się wysoką wydajnością oraz niskim narzutem czasowym, co sprawia, że jest jedną z najszybszych opcji w ekosystemie\cite{actix}.

Framework Actix-web został wybrany ze względu na:
\begin{itemize}[itemsep=0.2em, parsep=0pt, topsep=0pt]
	\item wysoką wydajność potwierdzoną testami benchmarkowymi,
	\item dobrą integrację z ekosystemem Rust,
	\item wsparcie dla programowania asynchronicznego.
\end{itemize}

\begin{figure}[h]
	\centering
	\includegraphics[width=0.9\textwidth]{images/rust_actix_web}
	\caption{Framework Actix-web w porównaniu z innymi frameworkami.}
	\label{fig:rust_actix_web}
\end{figure}

Rysunek~\ref{fig:rust_actix_web} ilustruje doskonałe wyniki wydajnościowe frameworka Actix w zakresie obsługi intensywnego ruchu – cecha szczególnie istotna w aplikacjach wymagających minimalnych czasów odpowiedzi. REST API zaimplementowano zgodnie z nowoczesnymi wzorcami programowania opisanymi przez Luca Palmieriego\cite{zero_to_prod}.

\section{Framework Tonic}

\textbf{Tonic} jest frameworkiem do implementacji gRPC w języku Rust, opartym na bibliotece \texttt{tokio} oraz protokole HTTP/2\cite{tonic}. W projekcie został wykorzystany do:
\begin{itemize}[itemsep=0.2em, parsep=0pt, topsep=0pt]
	\item implementacji serwera gRPC,
	\item definiowania kontraktów komunikacyjnych w postaci plików \texttt{.proto},
	\item realizacji wydajnej komunikacji klient–serwer.
\end{itemize}

Framework Tonic umożliwił stworzenie stabilnej i szybkiej komunikacji typu RPC, co było kluczowe dla testów wydajnościowych i porównawczych z GraphQL.

\section{tungstenite-rs}

\textbf{tungstenite-rs} jest biblioteką w Rust umożliwiającą obsługę protokołu WebSocket. Została wykorzystana do\cite{tungstenite}:
\begin{itemize}[itemsep=0.2em, parsep=0pt, topsep=0pt]
	\item obsługi komunikacji w czasie rzeczywistym,
	\item przesyłania danych bez konieczności inicjowania nowych połączeń HTTP,
	\item testów alternatywnych modeli komunikacji.
\end{itemize}

\section{h2load}

\texttt{h2load} jest narzędziem służącym do testowania wydajności aplikacji korzystających z protokołu HTTP/2\cite{h2load}. W projekcie zostało wykorzystane do:
\begin{itemize}[itemsep=0.2em, parsep=0pt, topsep=0pt]
	\item generowania dużej liczby równoległych zapytań,
	\item pomiaru czasów odpowiedzi serwera,
	\item analizy zachowania systemu pod obciążeniem.
\end{itemize}

\section{Python}

\textbf{Python} jest językiem wysokiego poziomu, który w projekcie został użyty głównie pomocniczo, do:
\begin{itemize}[itemsep=0.2em, parsep=0pt, topsep=0pt]
	\item analizy wyników testów wydajnościowych,
	\item generowania wykresów porównawczych,
	\item przetwarzania danych pomiarowych.
\end{itemize}

Python pozostaje jednym z najpopularniejszych języków programowania\cite{python}, co obrazuje Rysunek~\ref{fig:python_tiobe}.

\begin{figure}[h]
	\centering
	\includegraphics[width=0.8\textwidth]{images/python_tiobe_index}
	\caption{Pozycja języka Python w rankingu Tiobe Index.}
	\label{fig:python_tiobe}
\end{figure}

\section{GraphQL}

\textbf{GraphQL} jest językiem zapytań do API, pozwalającym klientowi precyzyjnie określić, jakie dane są potrzebne. W projekcie został wykorzystany do:
\begin{itemize}[itemsep=0.2em, parsep=0pt, topsep=0pt]
	\item realizacji zapytań o złożone struktury danych,
	\item testów wydajnościowych i porównawczych.
\end{itemize}

\section{gRPC}

\textbf{gRPC} jest mechanizmem komunikacji RPC opartym na protokole HTTP/2 oraz formacie binarnym Protocol Buffers. W projekcie zostało wykorzystane do:
\begin{itemize}[itemsep=0.2em, parsep=0pt, topsep=0pt]
	\item komunikacji między komponentami systemu,
	\item przeprowadzania testów wydajnościowych.
\end{itemize}

\section{WebAssembly}
\textbf{WebAssembly} (Wasm) to binarny format wykonywalny, który umożliwia natywne uruchamianie kodu Rust bezpośrednio w przeglądarce internetowej. Proces integracji WebAssembly z JavaScriptem został przeprowadzony zgodnie z metodologią przedstawioną w publikacji\cite{wasm_book}. 

Technologia ta została wykorzystana w projekcie do realizacji następujących celów:
\begin{itemize}[itemsep=0.2em, parsep=0pt, topsep=0pt]
	\item delegowania wybranych elementów logiki aplikacyjnej do wykonania po stronie klienta,
	\item przeprowadzenia eksperymentalnej analizy komparatywnej wydajności w odniesieniu do tradycyjnego JavaScriptu,
	\item optymalizacji parametrów wydajnościowych aplikacji webowej.
\end{itemize}

\section{rumqttc}

\textbf{rumqttc} jest asynchroniczną biblioteką w języku Rust służącą do obsługi protokołu MQTT (Message Queuing Telemetry Transport). Umożliwia ona implementację klientów MQTT, oferując niskie opóźnienia oraz wysoką wydajność komunikacji\cite{rumqttc}.

W projekcie biblioteka \texttt{rumqttc} została wykorzystana do:
\begin{itemize}[itemsep=0.2em, parsep=0pt, topsep=0pt]
	\item testów komunikacji typu publish–subscribe,
	\item analizy wydajności lekkich protokołów komunikacyjnych,
	\item porównania MQTT z innymi mechanizmami wymiany danych.
\end{itemize}

\section{wasm-pack}

\textbf{wasm-pack} jest narzędziem wspierającym proces kompilacji projektów napisanych w języku Rust do formatu WebAssembly. Automatyzuje on generowanie pakietów WASM, integrację z ekosystemem JavaScript oraz publikację modułów.

W projekcie \texttt{wasm-pack} został wykorzystany do:
\begin{itemize}[itemsep=0.2em, parsep=0pt, topsep=0pt]
	\item kompilacji modułów Rust do WebAssembly,
	\item przygotowania interfejsów JavaScript do komunikacji z WASM,
	\item budowy aplikacji demonstracyjnych uruchamianych w przeglądarce.
\end{itemize}

\section{Juniper}

\textbf{Juniper} jest biblioteką w języku Rust służącą do implementacji serwerów GraphQL. Umożliwia definiowanie schematów, zapytań oraz mutacji w sposób silnie typowany, zgodny z paradygmatami języka Rust\cite{juniper}.

W projekcie biblioteka \texttt{Juniper} została użyta do:
\begin{itemize}[itemsep=0.2em, parsep=0pt, topsep=0pt]
	\item implementacji API GraphQL,
	\item realizacji zapytań o złożone struktury danych,
	\item testów wydajnościowych komunikacji GraphQL.
\end{itemize}

\section{Apache Avro}

\textbf{Apache} Avro jest binarnym formatem serializacji danych opartym na schematach, wykorzystywanym głównie w systemach przetwarzania danych i architekturach rozproszonych. Zapewnia kompaktową reprezentację danych oraz możliwość ewolucji schematów.

W projekcie Apache Avro został wykorzystany do:
\begin{itemize}[itemsep=0.2em, parsep=0pt, topsep=0pt]
	\item testów wydajności serializacji i deserializacji danych,
	\item porównania z innymi formatami binarnymi i tekstowymi,
	\item analizy rozmiaru wynikowych struktur danych.
\end{itemize}

\section{BSON}

\textbf{BSON} (Binary JSON) jest binarną reprezentacją formatu JSON, zaprojektowaną z myślą o efektywnej serializacji danych. Jest powszechnie stosowany m.in. w bazach danych MongoDB.

W projekcie format BSON został wykorzystany do:
\begin{itemize}[itemsep=0.2em, parsep=0pt, topsep=0pt]
	\item testów wydajności serializacji danych,
	\item porównania binarnych i tekstowych formatów wymiany danych,
	\item analizy narzutu pamięciowego.
\end{itemize}

\section{Serde}

\textbf{Serde} jest biblioteką w języku Rust służącą do serializacji i deserializacji struktur danych\cite{serde}. Zapewnia zunifikowany interfejs obsługujący wiele formatów, takich jak JSON, BSON, XML czy MessagePack.

W projekcie \texttt{Serde} została wykorzystana jako:
\begin{itemize}[itemsep=0.2em, parsep=0pt, topsep=0pt]
	\item warstwa abstrakcji dla procesów serializacji danych,
	\item wspólna baza dla testów różnych formatów danych,
	\item narzędzie upraszczające implementację testów porównawczych.
\end{itemize}

\section{Prost}

\textbf{Prost} jest biblioteką w języku Rust służącą do obsługi formatu Protocol Buffers. Umożliwia generowanie kodu Rust na podstawie plików \texttt{.proto} oraz efektywną serializację danych binarnych\cite{prost}.

W projekcie \texttt{Prost} została wykorzystana do:
\begin{itemize}[itemsep=0.2em, parsep=0pt, topsep=0pt]
	\item obsługi komunikacji gRPC,
	\item serializacji danych w formacie Protocol Buffers,
	\item testów wydajności komunikacji RPC.
\end{itemize}

\section{quick-xml}

\textbf{quick-xml} jest szybką biblioteką w języku Rust przeznaczoną do parsowania i generowania dokumentów XML. Charakteryzuje się niskim narzutem pamięciowym oraz wysoką wydajnością\cite{qXML}.

W projekcie biblioteka \texttt{quick-xml} została użyta do:
\begin{itemize}[itemsep=0.2em, parsep=0pt, topsep=0pt]
	\item testów serializacji i deserializacji danych w formacie XML,
	\item porównania XML z innymi formatami wymiany danych,
	\item analizy wydajności przetwarzania danych tekstowych.
\end{itemize}